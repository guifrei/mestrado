\subsection{Estimativa da condutância térmica de contato}

No tratado sobre funções de Bessel escrito por \cite{livro_bessel}, são deduzidas soluções para problemas de propagação de ondas e de fluxo de
eletricidade que envolvem essas funções. No início do capítulo XII, os autores afirmam que é possível fazer uma analogia entre os problemas
elétricos desenvolvidos no texto com problemas de condução de calor, desde que se relacione, por exemplo, o potencial elétrico com a temperatura. Algumas
dessas análises teóricas levam em conta a resistência elétrica na região de contato entre dois materiais, considerando a existência de um fino
estrato nessa região. Os autores pressupõem a presença de uma descontinuidade de potencial elétrico entre as superfícies,
que seria função da resistividade do material do estrato, e resolvem analiticamente o problema difusivo de potencial elétrico nessa região de pequena espessura. 
Com base nisso, \cite{tese_mikic} afirma que, ainda de forma indireta, as primeiras análises teóricas sobre o fluxo de calor através de superfícies
em contato remontam à época de publicação desse trabalho. Mesmo assim, os primeiros estudos efetivos acerca da estimativa da CTC eram basicamente experimentais,
e foram impulsionados por necessidades nas áreas de projeto de reatores nucleares e na indústria aeroespacial \citep{tese_mikic}.

\cite{artigo_fenech} apresentam um modelo
teórico de CTC, baseado na hipótese de o contato entre as superfícies metálicas acontecer em apenas em alguns pontos, cuja distribuição superficial
é uniforme, e desprezando o fluxo de calor
nas regiões de vazios. A expressão analítica da CTC encontrada depende principalmente de três parâmetros: número de pontos de contato por unidade
de área, altura média dos vazios e razão entre área efetiva de contato e área total. Esses parâmetros são obtidos a partir de medidas tomadas nas
interfaces individuais, com auxílio de um perfilômetro. Os autores comparam os resultados teóricos com observações experimentais, e observam que,
a baixas pressões, a CTC se aproxima da condutividade térmica do fluido que preenche os vazios da interface. É importante observar que neste trabalho
e nos seguintes, a CTC é calculada como um valor único e constante sobre a superfície de contato.

Uma situação especialmente interessante para a área aeroespacial é o estudo da CTC em ambientes no vácuo. Este caso foi examinado por \cite{artigo_clausing}.
Em seu modelo, os autores distinguem os efeitos de constrição microscópica e macroscópica, e afirmam que estes últimos têm influência preponderante
na resistência térmica de contato. Tambem comentam sobre o efeito da chamada ``resistência de filme'', causado, por exemplo, pela presença de
filmes de óxidos em superfícies metálicas, e que podem apresentar influência significativa em ambientes no vácuo, sendo normalmente desprezados
nas análises em que há fluido presente no interstício. São definidos dois parâmetros adimensionais, baseados nas características geométricas e
físicas dos materiais, a partir dos quais é definida a CTC. 

\cite{tese_mikic} levantou expressões analíticas para resistências térmicas nos pontos de contato entre corpos metálicos, havendo presença de
fluido ou em ambiente de vácuo. Em seguida, fez uma descrição do perfil de rugosidade das superfícies de contato
a partir da distribuição gaussiana de probabilidade, a fim de determinar o número de contatos por unidade de área. Finalmente, relacionou a área efetiva de 
contato com a pressão aplicada sobre os materiais. Efeitos de filme foram desprezados, de forma a considerar apenas os efeitos de constrição.
Neste trabalho, a fase experimental, para fins de comparação com o modelo teórico, envolveu o emprego de um analisador de superfície para registrar os perfis de superfície, a partir dos quais foram obtidos
os desvios-padrão das amplitudes das rugosidades. A concordância entre os valores de CTC calculados e medidos foi, em geral, satisfatória; ainda assim,
para o caso de fluidos de baixa condutividade térmica, os valores de CTC previstos pelo modelo foram maiores do que os obtidos através dos experimentos. 

Uma investigação do efeito da difusão de calor em regime transiente na CTC foi conduzida por \cite{artigo_beck}. Nesse artigo, os campos de temperatura são
calculados separadamente para cada um dos dois corpos em contato, e medidas de temperatura são tomadas em pontos não necessariamente próximos da interface
de contato. A estimativa da CTC é a que miminiza o erro quadrático médio entre as temperaturas medidas e calculadas nesses pontos, avaliado em cada instante
discreto de tempo em que são feitas as medições.

Vários experimentos envolvendo fluidos diferentes na interface entre os corpos em contato, e sua influência na CTC, foram realizados por \cite{artigo_madhusudana}.
Considerações teóricas não foram feitas nesse trabalho. O autor mostra empiricamente que a presença de um meio condutor de calor na interface melhora a CTC,
especialmente a baixas pressões e se o meio for um bom condutor.   

A influência na CTC da formação de filmes de óxido em contatos metal-metal foi examinada por \cite{artigo_astrabadi}. Na modelagem, assumiu-se
que cada microcontato era circundado por uma região anular de óxido; desse modo a CTC total era uma composição entre duas parcelas, uma devido
aos contatos metal-metal circundados por óxido, e outra devido aos contatos óxido-óxido. Era esperado um aumento da resistência térmica de contato
devido à presença do filme, o que foi corroborado pelos resultados experimentais. O modelo envolve um descrição probabilística gaussiana da rugosidade e
da aspereza (coeficiente angular das microdeformações, aproximadas por cones).

\cite{artigo_snaith} citam a importância do razoável conhecimento do comportamento da transferência de calor no contato entre superfícies, citando
exemplos nas áreas de energia nuclear, microeletrônica e criogenia. Os autores fazem uma revisão dos trabalhos teóricos e empíricos sobre CTC
até o momento, apontando algumas limitações associadas às predições analíticas, tais como a suposição de que o contato é isotérmico, ou que as linhas de fluxo
de calor são paralelas a grandes distâncias da superficie de contato. São feitas comparações de predições de diferentes correlações empíricas
para a CTC em duas configurações hipotéticas (contatos entre ligas de alumínio e contatos entre elementos de aço inoxidável), mostrando
consideráveis divergências entre essas correlações.

Alguns experimentos foram realizados por \cite{artigo_williamson}, numa tentativa de responder algumas questões acerca da dependência da CTC com
deformações plásticas ou elásticas das superfícies em contato. Nos ensaios, foram aplicadas cargas cíclicas sobre os materiais em contato, e observou-se
uma influência significativa da deformação plástica apenas no primeiro carregamento. Um comparativo envolvendo modelos teóricos de CTC que levam em conta
o tipo de deformação foi feito por \cite{artigo_mcwaid}, e os autores observaram que as predições baseadas
no modelo de contato elástico concordavam melhor com resultados experimentais do que os baseados no modelo de contato plástico. 

Uma proposta de modelagem mecânica-geométrica foi apresentada por \cite{artigo_salgon}. Nesse modelo, a resistência térmica de contato é considerada
como a associação em paralelo de duas resistências: uma devido aos pontos de contato e outra devido ao meio intersticial. Os pontos de contato foram
representados segundo um modelo geométrico simplificado denominado \textit{tubo de Holm} \citep{livro_holm}. O cálculo da CTC apresentado no trabalho
depende de um conjunto de parâmetros adimensionais, calculados a partir das caraterísticas mecânicas e geométricas dos materiais envolvidos. Para baixos carregamentos,
os valores previstos pelo modelo diferiam de forma considerável dos resultados experimentais.

\cite{artigo_tomimura} introduziram uma abordagem baseada na transmissão de ondas sonoras através da interface de contato. Os autores definem a
taxa de transmissão de energia sonora como sendo a razão entre a energia transmitida com e sem atenuação na interface; a energia sonora, por sua
vez, é proporcional ao quadrado da variação de pressão detectada por transdutores localizados nas extremidades dos corpos de prova.
Essa taxa é relacionada com a condutância térmica de contato média sobre a interface, através de uma correlação bastante simples obtida a partir de dados
experimentais. Mesmo assim, o erro garantido pelos autores na previsão da CTC média é relativamente alto (aproximadamente $\pm 50 \%$).

Em face das dificuldades relacionadas ao levantamento de características me-cânicas e geométricas dos problemas de estimação de CTC, 
tais como a distribuição dos pontos de contato e as dimensões dos espaços intersticiais, e que são parâmetros necessários nos trabalhos citados
até o momento, não tardou para que começassem a ser aplicadas as técnicas de resolução de problemas inversos de transferência de calor (IHTP, do termo em inglês \textit{Inverse Heat Transfer Problems}).

\cite{artigo_huang} formularam um problema inverso de condução de calor em regime transiente a fim de estimar a condutância térmica no contato entre
os tubos e as aletas de um trocador de calor. Os autores resolveram numericamente o problema direto de condução de calor, atribuindo uma expressão
algébrica para a CTC em função do tempo e do ângulo (uma vez que a superfície de contato era circular), a fim de simular medidas de temperaturas
em determinadas posições. Em seguida, aplicaram o método do gradiente conjugado a fim de minimizar um funcional que envolvia valores calculados
de temperatura para uma estimativa de CTC e os valores medidos correspondentes, ao longo do tempo. Foram também realizadas simulações em que se
consideravam erros randômicos de distribuição gaussiana nas medições sintéticas. Os autores verificaram que, mesmo com a introdução de pequenos erros de medição e com
o aumento da distância dos pontos de medição em relação ao centro da interface circular, era possível obter estimativas confiáveis da CTC. O mesmo
problema foi formulado de forma não-linear e resolvido por \cite{artigo_huang_2}, gerando resultados ainda mais precisos.

\cite{artigo_milosevic} combinaram a técnica de estimação de parâmetros de Gauss com o método de pulso de \textit{laser} (originalmente empregado
na medição de difusividade térmica de materiais) a fim de estimar a CTC entre dois sólidos. Assim como no trabalho citado previamente, foram calculadas
medidas sintéticas de temperatura obtidas do problema direto, além de simulações envolvendo ruídos gaussianos adicionados a essas medidas.

A variação temporal da CTC também foi considerada por \cite{artigo_yang} ao estudar os efeitos presentes na interface entre um cabo de
fibra óptica e seu revestimento. De forma semelhante ao trabalho de \cite{artigo_huang}, foi aplicado o método do gradiente conjugado para estimação
da resistência térmica de contato nessa interface. Para a resolução do problema inverso, não foi necessário um conhecimento prévio da forma funcional das grandezas desconhecidas.

\cite{artigo_fieberg} relizaram investigações voltadas à análise da CTC de materiais aplicados em motores a combustão. Em seu trabalho, o salto de temperatura
na interface foi determinado através de medições obtidas com o uso de câmeras de infravermelho, e o fluxo de calor na interface foi calculado
através da resolução de um problema inverso de condução de calor; a razão entre essas grandezas forneceu a estimativa da CTC, de acordo
com a definição.   

Um problema inverso de condução de calor para estimativa de CTC entre corpos cujo contato varia ciclicamente foi analisado por \cite{artigo_shoj}.
Os autores levantaram duas classes de resultados, uma obtida por dados simulados com erros gaussianos, e outra com dados experimentais, obtidos através
de termopares poisicionados ao longo dos corpos de prova. O problema
inverso foi resolvido de forma iterativa para ambas as classes. Os valores de salto de temperatura na interface, necessários para o cálculo da CTC,
foram determinados de forma direta na primeira classe, e determinados através de regressão linear na segunda classe. 

A variação espacial da CTC ao longo da interface de contato foi considerada por \cite{artigo_gill}, que, na resolução do problema inverso, aplicaram
o método dos elementos de contorno \citep{livro_bem} associado a um algoritmo genético para regularização das medidas de temperatura. O procedimento empregado
envolvia a medição de temperaturas na interface. Uma simulação numérica do experimento foi implementada, sendo observada uma razoável sensibilidade
da solução em relação aos erros de medição. 

Duas características comums aos métodos até agora citados para determinação da CTC são: a) a necessidade de ter medidas de temperatura disponíveis na interface de contato,
normalmente por meios intrusivos, ou seja, medidas diretas de temperatura no interior dos corpos de prova; b) a necessidade de se fazer alguma descrição física e geométrica, em
escala microscópica, da interface de contato, o que pode exigir a separação dos materiais. Em face dessas dificuldades, \cite{reciproc_3} desenvolveram
um método não-intrusivo e não-destrutivo para estimativa da distribuição espacial da CTC em regime permanente. Para tanto, a fim de resolver o problema inverso de condução de calor, os autores partiram do conceito de funcional de reciprocidade, originalmente
empregado na identificação de falhas ou descontinuidades planas em materiais \citep{artigo_andrieux}.

A metodologia empregada por \cite{reciproc_3}, e que basicamente foi replicada nos trabalhos posteriores envolvendo funcionais de reciprocidade para estimativa
da CTC, consiste em duas etapas. Na primeira,
são formulados dois problemas difusivos auxiliares, no mesmo domínio do problema original; um problema é relacionado ao salto de temperatura na interface, e
o outro é relacionado ao fluxo de calor na interface. As soluções desses problemas são usadas para obter dois conjuntos de funções ortogonais; dessa forma,
o salto de temperatura e o fluxo de calor são expressos como combinações lineares das funções ortogonais correspondentes. A etapa seguinte é a determinação 
dos coeficientes dessas combinações lineares através dos funcionais de reciprocidade, que consistem em integrais calculadas nas fronteiras do domínio, envolvendo as soluções dos
problemas auxiliares resolvidos previamente, além de medidas externas de temperatura e fluxo de calor. As integrais são tais que
se anulam ao longo da interface de contato, sendo calculadas apenas nas fronteiras externas dos corpos de prova, sobre as quais conhecem-se as temperaturas
ou são impostas condições de contorno. Finalmente, obtidos os coeficientes das expansões lineares, são calculadas as estimativas de fluxo de calor e de salto de temperatura na interface;
a CTC é então avaliada através da razão entre essas quantidades. É importante destacar que a determinação da CTC por esta técnica é feita de forma não iterativa,
através da aplicação direta da definição. Outra vantagem desta técnica é que, para uma dada configuração geométrica dos corpos em contato, os problemas auxiliares que
fornecem as funções ortogonais só precisam ser resolvidos uma vez; desse modo, é possível estimar diferentes distribuições de CTC conhecendo-se apenas as medidas externas de temperatura
e de fluxo de calor, as quais, juntamente com as funções auxiliares, permitem calcular os funcionais de reciprocidade.

Os problemas auxiliares levantados por \cite{reciproc_3} são problemas de Cauchy, isto é, se caracterizam por terem duas condições de contorno sobre uma
mesma região da fronteira, o que requer o uso de técnicas especiais de solução. Os autores propuseram o método das soluções fundamentais \citep{artigo_marin}, aplicando-o
para seis diferentes perfis de CTC ao longo de uma interface de contato plana: um perfil constante ao longo da interface, dois perfis com variações suaves e três perfis apresentando descontinuidades. Para os
perfis sem descontinuidades (os três primeiros descritos acima), os resultados obtidos foram muito bons, mesmo após o acréscimo de erros gaussianos às medições
de temperatura. Para os casos de perfis descontínuos, os resultados permitiram recuperar o comportamento variacional das descontinuidades ao longo da interface.

\cite{artigo_colaco_2} estenderam o trabalho citado anteriormente, introduzindo um termo transiente no problema inverso de determinação da CTC, mantendo a mesma formulação
nos problemas auxiliares. A
metodologia adotada foi a mesma, incluindo o emprego da técnica das soluções fundamentais. Assim, chegaram a uma relação envolvendo os funcionais de reciprocidade,
que, no regime transiente, se reduzia ao resultado encontrado no trabalho anterior. Desse modo, a CTC encontrada era função do tempo, e sua estimativa era calculada 
iterativamente, melhorando à medida em que se aproximava do regime permanente.

Uma nova abordagem do tratamento transiente foi feita por \cite{artigo_colaco_4},
ao introduzir termos transientes nos problemas auxiliares, ainda aplicando a mesma metodologia baseada em funcionais de reciprocidade e soluções fundamentais. O tratamento
matemático envolvia a integração das funções de reciprocidade no tempo, eliminando o aspecto iterativo observado no trabalho anterior. Os resultados encontrados mostraram
as mesmas características dos trabalhos anteriores, indicando assim a robustez do método apresentado.

O método dos funcionais de reciprocidade foi empregado por \cite{tese_abreu} e \cite{artigo_abreu_3} na detecção indireta de falhas de contato planas em materiais bicompostos, através da estimativa da distribuição espacial da CTC.
Em situações práticas, tal falha é inacessível, e o método proposto se adequa por não ser intrusivo. Um aparato experimental, com descontinuidades na interface produzidas artificialmente, e a partir
do qual foram obtidas medidas reais de temperatura, foi construído para validar a modelagem. Para resolução dos problemas auxiliares, foi empregado o método 
de Monte Carlo com Cadeias de Markov \citep{artigo_mcmc}. A partir da estimativa do perfil da CTC, foi possível identificar e caracterizar qualitativamente as regiões de falha na interface,
de forma satisfatória.

\cite{artigo_padilha_3} avançaram no desenvolvimento do método dos funcionais de reciprocidade na estimativa da CTC ao aplicarem a reformulação das condições de contorno 
dos problemas auxiliares proposta por \cite{tese_abreu}, convertendo-os em problemas de valor de contorno e possibilitando o uso da Técnica da Transformada Integral Generalizada \citep{livro_integral_transforms_cotta}.
Uma definição conveniente das condições de contorno permitiu que o sistema linear que forneceria os coeficientes das expansões lineares se transformasse num sistema diagonal, simplificando
consideravelmente o cálculo desses coeficientes. Com isso, 
o problema inverso originalmente explorado por \cite{reciproc_3} foi desenvolvido analiticamente, fornecendo uma equação simples para o cálculo da CTC. O ganho
computacional obtido através desta abordagem foi considerável, uma vez que a distribuição da CTC pôde ser obtida em frações de segundo, de forma direta, obtendo resultados consistentes
com os encontrados através dos funcionais de reciprocidade empregando outras técnicas de solução.


