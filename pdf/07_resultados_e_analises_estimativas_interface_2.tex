\subsubsection{Estimativas para a geometria de interface de contato 2}

A próxima geometria de interface de contato a ser analisada corresponde à de índice 2 na tabela \ref{tabela_interfaces}. Trata-se de uma curva polinomial definida por partes, ao longo do domínio $0 \le x \le a$, construída algebricamente de forma a garantir continuidade tanto na curva quanto na sua primeira derivada (cf. Figura \ref{figura_interfaces}). Os perfis estimados de salto de temperatura na interface de contato, correspondentes aos três perfis teóricos de CTC, podem ser visualizados na Figura \ref{figura_delta_temperaturas_interface_02}.
\begin{figure}[H]
	\graficoestimativa{delta_temperatura}{2}{1}{20}{10}{06}{a}{\Delta T\big|_{\Gamma}}{\celsius}
	\graficoestimativa{delta_temperatura}{2}{2}{20}{09}{02}{b}{\Delta T\big|_{\Gamma}}{\celsius}
	\graficoestimativa{delta_temperatura}{2}{3}{20}{09}{05}{c}{\Delta T\big|_{\Gamma}}{\celsius}
	\caption{Comparação entre as estimativas de $[T_1 - T_2]_\Gamma$ e os valores exatos para os perfis de CTC de 1 a 3, referentes à interface de contato 2: $\text{--} \rightarrow \text{Exato}$; $\textcolor{blue}{\ocircle} \rightarrow \sigma = 0,0$; $\textcolor{red}{\square} \rightarrow \sigma = 0,1$; $\textcolor{gray}{\triangle} \rightarrow \sigma = 0,5$}
	\label{figura_delta_temperaturas_interface_02}
\end{figure}

A Tabela \ref{tabela_rms_delta_temperaturas_interface_2} mostra os valores de desvio quadrático médio entre as medidas estimadas e as teóricas de salto de temperatura para cada perfil teórico de condutância térmica de contato.
\begin{table}[H]
	\centering
	\caption{Desvio quadrático médio das estimativas de salto de temperatura para a interface de contato 2}
	\begin{tabular}{c|c|c|c|}
		\cline{2-4}
		& \multicolumn{3}{c|}{$\text{RMS}_{\Delta T} (\celsius)$} \\ \hline
		\multicolumn{1}{|c|}{Perfil} & $\sigma = \text{0,0}\celsius$   & $\sigma = \text{0,1}\celsius$    & $\sigma = \text{0,5}\celsius$  \\ \hline
		\multicolumn{1}{|c|}{1}      & 0,4128       & 0,8800       & 1,4149      \\ \hline
		\multicolumn{1}{|c|}{2}      & 0,3127       & 0,7087       & 1,0423      \\ \hline
		\multicolumn{1}{|c|}{3}      & 0,3672       & 0,7919       & 1,3930      \\ \hline
	\end{tabular}
	\label{tabela_rms_delta_temperaturas_interface_2}
\end{table}

Novamente é possível observar uma boa aderência aos perfis teóricos calculados via simulação do problema direto correspondente. Os perfis estimados para $\sigma = 0,0\celsius$ são os que melhor se aproximam dos perfis esperados, ainda que apresentem alguma dificuldade de convergência na extremidade direita do domínio.

Os gráficos da Figura \ref{figura_fluxo_calor_interface_02} mostram as estimativas para o fluxo de calor ao longo da interface de contato, correspondentes aos três perfis teóricos de CTC.

%
%\begin{figure}[h!b]
%	\graficoerrorms{norma}{delta_temperatura}{2}{1}{[T_1 - T_2]_\Gamma}{a}
%	\graficoerrorms{norma}{delta_temperatura}{2}{2}{[T_1 - T_2]_\Gamma}{b}
%	\graficoerrorms{norma}{delta_temperatura}{2}{3}{[T_1 - T_2]_\Gamma}{c}
%	\caption{Erro $\log[\text{RMS}(\delta)]$ das estimativas de $[T_1 - T_2]_\Gamma$ versus o número de funções ortonormais ($N_j$), para o perfis de CTC de 1 a 3, referentes à interface de contato 2: $\textcolor{blue}{\ocircle} \rightarrow \sigma = 0,0$; $\textcolor{red}{\square} \rightarrow \sigma = 0,1$; $\textcolor{gray}{\triangle} \rightarrow \sigma = 0,5$}
%\end{figure}
%
\begin{figure}[H]
	\graficoestimativa{fluxo_calor}{2}{1}{20}{06}{02}{a}{-k_1 \frac{\partial T_1}{\partial\mathbf{n}_1}\big|_{\Gamma}}{W/$\text{m}^2$}
	\graficoestimativa{fluxo_calor}{2}{2}{20}{04}{04}{b}{-k_1 \frac{\partial T_1}{\partial\mathbf{n}_1}\big|_{\Gamma}}{W/$\text{m}^2$}
	\graficoestimativa{fluxo_calor}{2}{3}{20}{06}{03}{c}{-k_1 \frac{\partial T_1}{\partial\mathbf{n}_1}\big|_{\Gamma}}{W/$\text{m}^2$}
	\caption{Comparação entre as estimativas de $\left[-k_1 \frac{\partial T_1}{\partial\mathbf{n}_1}\right]_\Gamma$ e os valores exatos para os perfis de CTC de 1 a 3, referentes à interface de contato 2: $\text{--} \rightarrow \text{Exato}$; $\textcolor{blue}{\ocircle} \rightarrow \sigma = 0,0$; $\textcolor{red}{\square} \rightarrow \sigma = 0,1$; $\textcolor{gray}{\triangle} \rightarrow \sigma = 0,5$}
	\label{figura_fluxo_calor_interface_02}
\end{figure}

A Tabela \ref{tabela_rms_fluxo_calor_interface_2} mostra os valores de desvio quadrático médio entre as medidas estimadas e as teóricas de fluxo de calor para cada perfil teórico de condutância térmica de contato.
\begin{table}[H]
	\centering
	\caption{Desvio quadrático médio das estimativas de fluxo de calor para a interface de contato 2}
	\begin{tabular}{c|c|c|c|}
		\cline{2-4}
		& \multicolumn{3}{c|}{$\text{RMS}_{q}(\text{W/m}^2)$} \\ \hline
		\multicolumn{1}{|c|}{Perfil} & $\sigma = \text{0,0}\celsius$   & $\sigma = \text{0,1}\celsius$    & $\sigma = \text{0,5}\celsius$  \\ \hline
		\multicolumn{1}{|c|}{1}      & 1290,26          &  2552,23      &  3324,56      \\ \hline
		\multicolumn{1}{|c|}{2}      & 84,44       & 668,69       & 1141,60      \\ \hline
		\multicolumn{1}{|c|}{3}      & 1008,87       & 1947,80       & 2349,65      \\ \hline
	\end{tabular}
	\label{tabela_rms_fluxo_calor_interface_2}
\end{table}

Novamente a solução para $\sigma = 0,0\celsius$ é a que apresenta melhor concordância com o comportamento teórico esperado, enquanto as soluções encontradas para os outros valores de desvio-padrão possuam características qualitativas compatíveis com o esperado. O efeito Gibbs também pode ser observado nas estimativas de fluxo de calor referentes aos perfis teóricos descontínuos de CTC.

%
%\begin{figure}[h!b]
%	\graficoerrorms{erro_rms}{fluxo_calor}{2}{1}{[-k_1 \partial T_1/\partial\mathbf{n}]_\Gamma}{a}
%	\graficoerrorms{erro_rms}{fluxo_calor}{2}{2}{[-k_1 \partial T_1/\partial\mathbf{n}]_\Gamma}{b}
%	\graficoerrorms{erro_rms}{fluxo_calor}{2}{3}{[-k_1 \partial T_1/\partial\mathbf{n}]_\Gamma}{c}
%	\caption{Erro $\log[\text{RMS}(\delta)]$ das estimativas de $[-k_1 \partial T_1/\partial\mathbf{n}]_\Gamma$ versus o número de funções ortonormais ($N_j$), para o perfis de CTC de 1 a 3, referentes à interface de contato 2: $\textcolor{blue}{\ocircle} \rightarrow \sigma = 0,0$; $\textcolor{red}{\square} \rightarrow \sigma = 0,1$; $\textcolor{gray}{\triangle} \rightarrow \sigma = 0,5$}
%\end{figure}
%

Calculando-se a razão entre o fluxo de calor e o salto de temperatura em cada ponto do domínio da interface, exatamente como foi feito na interface analisada na subseção anterior, obtém-se as estimativas dos perfis de condutância térmica de contato, representadas graficamente na Figura \ref{figura_ctc_interface_02}.
\begin{figure}[H]
	\graficoctc{02}{01}{1}{a}
	\graficoctc{02}{02}{2}{b}
	\graficoctc{02}{03}{3}{c}
	\caption{Comparação entre as estimativas de $h_c$ e os valores exatos para os perfis de CTC de 1 a 3, referentes à interface de contato 2: $\text{--} \rightarrow \text{Exato}$; $\textcolor{blue}{\ocircle} \rightarrow \sigma = 0,0$; $\textcolor{red}{\square} \rightarrow \sigma = 0,1$; $\textcolor{gray}{\triangle} \rightarrow \sigma = 0,5$}
	\label{figura_ctc_interface_02}
\end{figure}

A Tabela \ref{tabela_rms_ctc_interface_2} mostra os valores de desvio quadrático médio entre as medidas estimadas e as teóricas de condutância térmica de contato, para os três perfis analisados. A razão entre cada desvio quadrático médio e o valor máximo da condutância térmica de contato teórica ($h_{max} = 400 \text{ W/m}^2$ \celsius) varia de 0,008075 a 0,219025.
\begin{table}[H]
	\centering
	\caption{Desvio quadrático médio das estimativas de condutância térmica de contato para a interface de contato 2}
	\begin{tabular}{c|c|c|c|}
		\cline{2-4}
		& \multicolumn{3}{c|}{$\text{RMS}_{h_c}(\text{W/m}^{2}$$\celsius)$} \\ \hline
		\multicolumn{1}{|c|}{Perfil} & $\sigma = \text{0,0}\celsius$   & $\sigma = \text{0,1}\celsius$    & $\sigma = \text{0,5}\celsius$  \\ \hline
		\multicolumn{1}{|c|}{1}      & 40,79       &  71,12      & 87,61      \\ \hline
		\multicolumn{1}{|c|}{2}      & 3,23       & 22,33       &  40,37     \\ \hline
		\multicolumn{1}{|c|}{3}      & 37,35       & 64,35       & 82,18      \\ \hline
	\end{tabular}
	\label{tabela_rms_ctc_interface_2}
\end{table}

De forma semelhante ao caso anterior, a estimativa de CTC calculada para $\sigma = 0,0\celsius$ foi a que melhor se aproximou do perfil teórico de CTC, apresentando inclusive resultados muito parecidos com os daquele caso. Os perfis de CTC estimados para os outros níveis de ruído mantiveram o padrão qualitativo de comportamento observado nos perfis estimados na subseção anterior.