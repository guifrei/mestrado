\section{Revisão bibliográfica}

A revisão bibliográfica exposta nesta seção está dividida em duas partes. Na primeira parte, é feito um levantamento sobre a evolução histórica
das técnicas de solução de problemas difusivos, uma vez que, conforme será visto nos próximos capítulos, o problema inverso de condução de calor
a partir do qual será estimada a CTC é um problema difusivo. Na segunda parte, é apresentado um apanhado dos trabalhos clássicos que envolvem o tema da
determinação da CTC, antes e depois da introdução da abordagem deste tema como um problema inverso.

\subsection{Técnicas de soluções de problemas difusivos}

Segundo \cite{livro_tanehill}, há basicamente três abordagens ou métodos que podem ser usados para resolver um problema
de mecânica dos fluidos e/ou transferência de calor: experimental, teórico (ou analítico) e computacional (ou numérico). O primeiro
fornece resultados mais realistas a um custo de implementação maior. Já o segundo faz suposições
simplificadoras a fim de facilitar o tratamento do problema, e possivelmente encontrar uma solução fechada, ou seja, que é expressa geralmente como uma fórmula
matemática; uma vez que não envolve iterações ou interpolações, a obtenção do valor da solução num determinado ponto do domínio é praticamente imediata,
com baixo custo computacional. No último método -- a abordagem numérica -- as equações que governam o fenômeno em estudo são substituídas por
esquemas numéricos, cuja solução (obtida através de cálculos manuais ou pelo emprego de computadores digitais) é usada para representar de forma
aproximada a solução do problema original.

A teoria matemática moderna sobre a condução de calor foi estabelecida por Joseph Fourier, que reuniu suas investigações teóricas em sua obra
\textit{Théorie analytique de la chaleur} \citep{livro_fourier, artigo_langer}. Neste trabalho, Fourier deduz analiticamente as equações de condução para vários
tipos de sólidos, como esferas, prismas retangulares ou cilindros de seção reta circular. Nesses problemas, o fluxo de calor tinha
sempre uma direção que favorecia o aspecto simétrico do sólido em questão; por exemplo, no caso do cilindro ou da esfera, o fluxo
de calor acontecia na direção radial. Em seguida, Fourier apresenta a formulação tridimensional em coordenadas cartesianas da equação
de condução de calor em regime transiente, e a emprega para levantar as equações obtidas previamente, através de substituições de
variáveis. Finalmente, Fourier desenvolve o conceito de representação de uma função como um somatório infinito de funções
trigonométricas\footnote{A história registra que, quando Fourier apresentou seus artigos sobre a representação de funções arbitrárias
como expansões de senos e cossenos à Academia de Ciências de Paris em 1807 e 1811, recebeu críticas dos consultores (principalmente Lagrange,
que negou veementemente essa possibilidade), devido à falta de rigor, e por isso os artigos não foram publicados \citep{livro_agarwal}.} e, através da aplicação
da técnica de separação de variáveis, usa esse conceito na resolução analítica dos problemas que propôs no início do trabalho. Uma vez que os domínios envolvidos e as
condições de contorno impostas nos problemas possuíam algum tipo de simetria, que simplificava a variação da temperatura para apenas uma
variável dimensional (distância radial, comprimento ao longo de um eixo, etc.), as soluções encontradas eram relativamente simples, o que não
diminui sua importância por modelarem matematicamente o fenômeno físico da condução de calor pela primeira vez. O tratamento
dado por Fourier aos problemas de condução de calor foi, em suma, predominantemente analítico, sem considerações de caráter numérico.

Historicamente, os livros acadêmicos que versam sobre transferência de calor reproduzem a dedução da equação de condução em coordenadas
cartesianas feita por Fourier e em seguida apresentam a formulação equivalente nos sistemas de coordenadas cilíndricas e esféricas; ver
por exemplo \cite{livro_carslaw}, \cite{livro_holman} e \cite{livro_ozisik}. Em tese, a formulação no sistema cartesiano é aplicável
a qualquer tipo de geometria; a solução geral sempre poderá ser expressa em termos de somatórios de senos ou cossenos. Soluções analíticas obtidas
nos sistemas de coordenadas esféricas ou cilíndricas envolvem somatórios de tipos diferentes de funções base ortogonais, obtidas através da aplicação do método
de separação de variáveis \citep{livro_boyce}. Voltando a citar os trabalhos de Fourier, ao expressar
as equações de condução para corpos esféricos ou cilíndricos, ele fez implicitamente um \textit{mapeamento} dessas geometrias a partir
do sistema de coordenadas cartesiano para os sistemas de coordenadas cilíndricas ou esféricas \citep{livro_numerical_grid}, e, ao aplicar a
separação de variáveis, encontra soluções em forma de séries que futuramente seriam associadas às funções de Bessel e Legendre \citep{livro_fourier}. 

\cite{artigo_einsenhart, artigo_einsenhart_2} demonstrou que a equação de Helmholtz (da qual a equação
de difusão é um caso particular) pode ser resolvida por separação de variáveis em onze diferentes
sistemas de coordenadas ortogonais, dentre as quais figuram os sistemas cartesiano, esférico e cilíndrico, bem como outros menos usuais, tais
como elíptico-cilíndrico ou parabólico. \cite{livro_moon} fizeram um extenso levantamento das equações diferenciais ordinárias
oriundas da separação de variáveis em cada um desses sistemas, e apresentam suas respectivas soluções gerais (dentre as quais estão as funções
seno e cosseno e as funções de Bessel e Legendre, já citadas). Tais equações, juntamente com as respectivas condições de contorno, constituem-se
em \textit{problemas de valor de contorno de Sturm-Liouville} \citep{artigo_sturm, artigo_liouville}. Problemas de Sturm-Liouville são na realidade
parte de uma teoria fundamental da Álgebra Linear conhecida como teoria dos operadores lineares; isto confere características e propriedades
especiais às soluções desses problemas (denominadas \textit{autofunções}), tais como a ortogonalidade e a possibilidade de, sob certas condições, expressar uma função arbitrária
como uma combinação linear dessas autofunções \citep{livro_boyce, livro_axler}.

Os métodos numéricos de solução de equações diferenciais ganharam grande impulso a partir da década de 1960, quando houve uma maior disponibilidade
de computadores digitais de alto desempenho \citep{livro_tanehill}. Aliado a esse fato, houve também um aumento natural da complexidade dos problemas de transferência
de calor e mecânica dos fluidos: adoção de geometrias complexas ou não convencionais, formulações de condições de contorno que introduziam dificuldades na
aplicação dos métodos analíticos, menores restrições quanto a não linearidade dos problemas. Já as abordagens numéricas, contudo, remontam a épocas anteriores. 

Uma dessas abordagens, considerada pioneira na análise numérica de equações diferenciais parciais em problemas
difusivos, foi proposta por \cite{artigo_richardson}. Em seu artigo, Richarsdon resolve numericamente a equação de Laplace e a equação biarmônica, e aplica esta última no problema prático de estudo da distribuição de tensão numa
barragem de alvenaria, com uma geometria bidimensional. Para tanto, ele representa o perfil da barragem usando segmentos de reta e emprega uma malha estruturada regular em seu interior. A malha
era resolvida usando um esquema iterativo de diferenças finitas centradas \citep{livro_tanehill}. Richardson já destacava as limitações dos métodos analíticos na integração de equações diferenciais parciais nos casos em que as fronteiras têm formato irregular.

Os anos seguintes testemunharam uma grande quantidade de pesquisas em métodos numéricos, especialmente no campo dos problemas de dinâmica dos fluidos. Avanços na área dos
problemas difusivos envolvem técnicas de precondicionamento de sistemas lineares e o desenvolvimento de esquemas de sobrerrelaxação, aumentando significativamente
o desempenho da taxa de convergência da solução \citep{artigo_frankel, artigo_fedorenko}.

Em meados da década de 1970, sistemas de coordenadas generalizadas, coincidentes com as fronteiras de domínios irregulares,
começam a ser empregados em detrimento dos sistemas de coordenadas ortogonais convencionais na resolução numérica de problemas advectivos-difusivos,
com o objetivo de evitar interpolações entre pontos da malha não coincidentes com as fronteiras \citep{livro_maliska}. As fronteiras do domínio físico, definidas geralmente no sistema 
cartesiano, são mapeadas para uma região retangular no domínio computacional, definido no sistema de coordenadas generalizado, através de relações de transformação.

A distribuição dos pontos na malha estruturada do problema físico original é feita de modo que a malha correspondente no domínio computacional seja regular. 
As equações diferenciais e as condições de contorno devem ser reescritas nesse novo sistema; as equações resultantes são mais complicadas uma vez que contêm mais
termos e coeficientes variáveis. Tais equações podem ser discretizadas e resolvidas usando, por exemplo, a técnica das diferenças finitas aplicada à malha retangular definida no domínio computacional.
Finalmente, a solução é transformada de volta para o domínio físico através das relações de transformação. Destaca-se o trabalho de \citet{artigo_thompson},
em que foi desenvolvida uma técnica para determinação da relação de transformação entre sistemas de coordenadas para o caso bidimensional através da resolução numérica de equações diferenciais parciais.

Outra técnica de transformação de coordenadas envolve o emprego de transformações conformes \citep{livro_numerical_grid}. Uma grande vantagem
dessa técnica é o fato de que a equação de difusão preserva a forma original, sem introdução de termos não-lineares ou derivadas cruzadas;
por outro lado é aplicável apenas a geometrias bidimensionais. Se as fronteiras da geometria do problema não tiverem uma descrição algébrica
conhecida, também deve se fazer uso de esquemas de geração numérica de malha, baseados por exemplo na transformação de Schwarz-Christoffel \citep{livro_brown}.

Voltando ao trabalho de Fourier, é possível identificar em seu trabalho uma das bases sobre a qual se estabeleceu a técnica analítica de solução
de equações diferenciais parciais conhecida como transformação integral. No tratado previamente citado \textit{Théorie analytique de la chaleur},
procurando estender suas ideias para funções definidas em intervalos infinitos, Fourier descobriu uma fórmula de transformação integral e
sua respectiva inversa, que hoje levam o seu nome \citep{livro_integral_transforms}.

A técnica da transformação integral revelou-se uma
ferramenta poderosa para resolver problemas de valor inicial e problemas de valor de contorno para equações diferenciais parciais lineares.
O conceito de transformação integral originou-se dos trabalhos de Laplace, que publicou os primeiros resultados envolvendo a sua transformada
no tratado de 1812, \textit{Théorie analytique des probabilités} \citep{livro_integral_transforms}. Cauchy publicou em 1843 uma descrição
dos métodos simbólicos ou operacionais (que trabalham com operadores diferenciais como se fossem algébricos) \citep{livro_cauchy} e sua relação com a transformada de
Laplace, além de apresentar a forma exponencial da transformada de Fourier. Contudo, quem popularizou o uso das transformadas de Laplace foi Oliver Heaviside,
aplicando-as na resolução de equações diferenciais lineares em problemas de circuitos elétricos, especialmente a equação do telégrafo, consolidando as bases do
cálculo operacional moderno \citep{livro_yavetz, artigo_carson}. 

A grande vantagem oferecida pela transformação integral aplicada às equações de difusão linear é a possibilidade de resolver analiticamente classes de
problemas para os quais a técnica de separação de variáveis não é adequada.
No caso específico das equações de difusão linear, as transformações integrais procuram reduzir a quantidade de variáveis independentes
na equação original, geralmente levando a uma nova equação diferencial ordinária de fácil solução. Exemplos elementares de aplicação
de transformação integral em problemas difusivos nos sistemas cartesiano e cilíndrico foram apresentados por \cite{artigo_doetsch}, \cite{artigo_sneddon} e \cite{livro_tranter},
e em coordenadas esféricas por \cite{artigo_olcer}.
\cite{livro_unified} classificaram e revisaram sete classes de problemas difusivos lineares, apresentando suas soluções
exatas obtidas pela técnica da transformação integral, que por razões históricas passou a ser denominada Técnica da Transformação Integral Clássica (CITT).

A CITT não se mostrou flexível o suficiente para fornecer
soluções analíticas para tipos de problemas mais gerais. Nesse sentido, o trabalho desenvolvido por \cite{artigo_murray},
que tratava problemas difusivos transientes com coeficientes variáveis nas condições de contorno, foi pioneiro no desenvolvimento da técnica
que é conhecida hoje como Transformada Integral Generalizada (GITT) \citep{livro_integral_transforms_cotta}. Desde então, a GITT --- uma técnica
híbrida analítico-numérica --- vem sido continuamente desenvolvida e estendida, fornecendo soluções analíticas aproximadas e numéricas alternativas
para os problemas não solucionáveis pela abordagem clássica. Nos últimos anos, a GITT tem sido beneficiada e popularizada através do desenvolvimento de
aplicativos de \textit{software} de computação simbólica, especialmente o \textit{Mathematica}\textsuperscript{\textregistered} \citep{artigo_mathematica}.

A técnica de transformação integral, seja clássica ou generalizada, é formulada com base em integrais sobre um domínio cuja
fronteira não é necessariamente regular. Em muitos casos, a transformação integral foi empregada em problemas difusivos ou advectivo-difusivos nos quais a geometria
era irregular, porém descrita de forma algébrica no sistema de coordenadas cartesiano, como por exemplo os trabalhos de \cite{artigo_aparecido, artigo_aparecido_2, artigo_aparecido_3, artigo_fausto} e \cite{artigo_perez}.
Já \cite{tese_sphaier} generaliza as técnicas de levantamento de autofunções nestes trabalhos e apresenta uma solução formal para domínios irregulares usando GITT;
para fronteiras cuja descrição algébrica é desconhecida, Sphaier sugere aproximar por linhas poligonais (no caso bidimensional), em que cada
trecho é representado por um segmento de reta. As autofunções levantadas no processo de solução são obtidas a partir de problemas auxiliares resolvidos no próprio
domínio irregular do problema, e nesse caso diz-se que o domínio é \textit{coincidente}; é feita uma rápida citação sobre soluções de problemas difusivos em domínios \textit{envolventes}
(quando o domínio original é considerado como sendo contido por um domínio maior, regular, sobre o qual a solução geral é encontrada e então particularizada
para aquele subdomínio).

As soluções da equação de difusão em dutos de seção reta elíptica com condições de contorno de primeiro e segundo tipo encontradas por
 \citet{trabalho_maia_1, trabalho_maia_2} combinaram as técnicas de transformação integral e mapeamento conforme. 
\cite{tese_antonini} usa a mesma metodologia em sua dissertação de mestrado sobre resolução de problemas difusivo-convectivos em geometrias não-convencionais (setor circular, geometria anular
concêntrica e geometria bicônica). Conforme citado anteriormente, as equações de difusão no novo sistema de coordenadas preservaram sua forma; isso facilitou em
muito o emprego da GITT, uma vez que os problemas de autovalor associado geraram funções seno e cosseno. Por outro lado, a
simplificação nas soluções analíticas ou híbridas encontradas só foi possível porque as geometrias envolvidas possuíam mapeamentos conformes de expressão analítica
conhecida \citep{livro_brown}.

Recentemente, houve o aumento do interesse na aplicação de métodos numéricos não baseados em malhas (\textit{meshless}) na resolução
de problemas difusivos em geometrias irregulares; tais métodos procuram superar as dificuldades computacionais envolvidas na geração
de malhas de discretização, trabalhando com o conceito de nós distribuídos pelo domínio do problema, e com as interações entre nós
vizinhos. Tais técnicas incluem, por exemplo, o emprego de redes neurais \citep{artigo_deng, artigo_heidari}, funções de base
radial associadas ao método de colocação \citep{artigo_chen, artigo_dai}, o método SPH (\textit{smoothed particle hydrodynamics}) \citep{artigo_vishwakarma} e o método MLPG (\textit{Meshless Local Petrov-Galerkin})
\citep{artigo_li}.

\subsection{Estimativa da condutância térmica de contato}

No tratado sobre funções de Bessel escrito por \cite{livro_bessel}, são deduzidas soluções para problemas de propagação de ondas e de fluxo de
eletricidade que envolvem essas funções. No início do capítulo XII, os autores afirmam que é possível fazer uma analogia entre os problemas
elétricos desenvolvidos no texto com problemas de condução de calor, desde que se relacione, por exemplo, o potencial elétrico com a temperatura. Algumas
dessas análises teóricas levam em conta a resistência elétrica na região de contato entre dois materiais, considerando a existência de um fino
estrato nessa região. Os autores pressupõem a presença de uma descontinuidade de potencial elétrico entre as superfícies,
que seria função da resistividade do material do estrato, e resolvem analiticamente o problema difusivo de potencial elétrico nessa região de pequena espessura. 
Com base nisso, \cite{tese_mikic} afirma que, ainda de forma indireta, as primeiras análises teóricas sobre o fluxo de calor através de superfícies
em contato remontam à época de publicação desse trabalho. Mesmo assim, os primeiros estudos efetivos acerca da estimativa da CTC eram basicamente experimentais,
e foram impulsionados por necessidades nas áreas de projeto de reatores nucleares e na indústria aeroespacial \citep{tese_mikic}.

\cite{artigo_fenech} apresentam um modelo
teórico de CTC, baseado na hipótese de o contato entre as superfícies metálicas acontecer em apenas em alguns pontos, cuja distribuição superficial
é uniforme, e desprezando o fluxo de calor
nas regiões de vazios. A expressão analítica da CTC encontrada depende principalmente de três parâmetros: número de pontos de contato por unidade
de área, altura média dos vazios e razão entre área efetiva de contato e área total. Esses parâmetros são obtidos a partir de medidas tomadas nas
interfaces individuais, com auxílio de um perfilômetro. Os autores comparam os resultados teóricos com observações experimentais, e observam que,
a baixas pressões, a CTC se aproxima da condutividade térmica do fluido que preenche os vazios da interface. É importante observar que neste trabalho
e nos seguintes, a CTC é calculada como um valor único e constante sobre a superfície de contato.

Uma situação especialmente interessante para a área aeroespacial é o estudo da CTC em ambientes no vácuo. Este caso foi examinado por \cite{artigo_clausing}.
Em seu modelo, os autores distinguem os efeitos de constrição microscópica e macroscópica, e afirmam que estes últimos têm influência preponderante
na resistência térmica de contato. Tambem comentam sobre o efeito da chamada ``resistência de filme'', causado, por exemplo, pela presença de
filmes de óxidos em superfícies metálicas, e que podem apresentar influência significativa em ambientes no vácuo, sendo normalmente desprezados
nas análises em que há fluido presente no interstício. São definidos dois parâmetros adimensionais, baseados nas características geométricas e
físicas dos materiais, a partir dos quais é definida a CTC. 

\cite{tese_mikic} levantou expressões analíticas para resistências térmicas nos pontos de contato entre corpos metálicos, havendo presença de
fluido ou em ambiente de vácuo. Em seguida, fez uma descrição do perfil de rugosidade das superfícies de contato
a partir da distribuição gaussiana de probabilidade, a fim de determinar o número de contatos por unidade de área. Finalmente, relacionou a área efetiva de 
contato com a pressão aplicada sobre os materiais. Efeitos de filme foram desprezados, de forma a considerar apenas os efeitos de constrição.
Neste trabalho, a fase experimental, para fins de comparação com o modelo teórico, envolveu o emprego de um analisador de superfície para registrar os perfis de superfície, a partir dos quais foram obtidos
os desvios-padrão das amplitudes das rugosidades. A concordância entre os valores de CTC calculados e medidos foi, em geral, satisfatória; ainda assim,
para o caso de fluidos de baixa condutividade térmica, os valores de CTC previstos pelo modelo foram maiores do que os obtidos através dos experimentos. 

Uma investigação do efeito da difusão de calor em regime transiente na CTC foi conduzida por \cite{artigo_beck}. Nesse artigo, os campos de temperatura são
calculados separadamente para cada um dos dois corpos em contato, e medidas de temperatura são tomadas em pontos não necessariamente próximos da interface
de contato. A estimativa da CTC é a que miminiza o erro quadrático médio entre as temperaturas medidas e calculadas nesses pontos, avaliado em cada instante
discreto de tempo em que são feitas as medições.

Vários experimentos envolvendo fluidos diferentes na interface entre os corpos em contato, e sua influência na CTC, foram realizados por \cite{artigo_madhusudana}.
Considerações teóricas não foram feitas nesse trabalho. O autor mostra empiricamente que a presença de um meio condutor de calor na interface melhora a CTC,
especialmente a baixas pressões e se o meio for um bom condutor.   

A influência na CTC da formação de filmes de óxido em contatos metal-metal foi examinada por \cite{artigo_astrabadi}. Na modelagem, assumiu-se
que cada microcontato era circundado por uma região anular de óxido; desse modo a CTC total era uma composição entre duas parcelas, uma devido
aos contatos metal-metal circundados por óxido, e outra devido aos contatos óxido-óxido. Era esperado um aumento da resistência térmica de contato
devido à presença do filme, o que foi corroborado pelos resultados experimentais. O modelo envolve um descrição probabilística gaussiana da rugosidade e
da aspereza (coeficiente angular das microdeformações, aproximadas por cones).

\cite{artigo_snaith} citam a importância do razoável conhecimento do comportamento da transferência de calor no contato entre superfícies, citando
exemplos nas áreas de energia nuclear, microeletrônica e criogenia. Os autores fazem uma revisão dos trabalhos teóricos e empíricos sobre CTC
até o momento, apontando algumas limitações associadas às predições analíticas, tais como a suposição de que o contato é isotérmico, ou que as linhas de fluxo
de calor são paralelas a grandes distâncias da superficie de contato. São feitas comparações de predições de diferentes correlações empíricas
para a CTC em duas configurações hipotéticas (contatos entre ligas de alumínio e contatos entre elementos de aço inoxidável), mostrando
consideráveis divergências entre essas correlações.

Alguns experimentos foram realizados por \cite{artigo_williamson}, numa tentativa de responder algumas questões acerca da dependência da CTC com
deformações plásticas ou elásticas das superfícies em contato. Nos ensaios, foram aplicadas cargas cíclicas sobre os materiais em contato, e observou-se
uma influência significativa da deformação plástica apenas no primeiro carregamento. Um comparativo envolvendo modelos teóricos de CTC que levam em conta
o tipo de deformação foi feito por \cite{artigo_mcwaid}, e os autores observaram que as predições baseadas
no modelo de contato elástico concordavam melhor com resultados experimentais do que os baseados no modelo de contato plástico. 

Uma proposta de modelagem mecânica-geométrica foi apresentada por \cite{artigo_salgon}. Nesse modelo, a resistência térmica de contato é considerada
como a associação em paralelo de duas resistências: uma devido aos pontos de contato e outra devido ao meio intersticial. Os pontos de contato foram
representados segundo um modelo geométrico simplificado denominado \textit{tubo de Holm} \citep{livro_holm}. O cálculo da CTC apresentado no trabalho
depende de um conjunto de parâmetros adimensionais, calculados a partir das caraterísticas mecânicas e geométricas dos materiais envolvidos. Para baixos carregamentos,
os valores previstos pelo modelo diferiam de forma considerável dos resultados experimentais.

\cite{artigo_tomimura} introduziram uma abordagem baseada na transmissão de ondas sonoras através da interface de contato. Os autores definem a
taxa de transmissão de energia sonora como sendo a razão entre a energia transmitida com e sem atenuação na interface; a energia sonora, por sua
vez, é proporcional ao quadrado da variação de pressão detectada por transdutores localizados nas extremidades dos corpos de prova.
Essa taxa é relacionada com a condutância térmica de contato média sobre a interface, através de uma correlação bastante simples obtida a partir de dados
experimentais. Mesmo assim, o erro garantido pelos autores na previsão da CTC média é relativamente alto (aproximadamente $\pm 50 \%$).

Em face das dificuldades relacionadas ao levantamento de características mecânicas e geométricas dos problemas de estimação de CTC, 
tais como a distribuição dos pontos de contato e as dimensões dos espaços intersticiais, e que são parâmetros necessários nos trabalhos citados
até o momento, não tardou para que começassem a ser aplicadas as técnicas de resolução de problemas inversos de transferência de calor (IHTP, do termo em inglês \textit{Inverse Heat Transfer Problems}).

\cite{artigo_huang} formularam um problema inverso de condução de calor em regime transiente a fim de estimar a condutância térmica no contato entre
os tubos e as aletas de um trocador de calor. Os autores resolveram numericamente o problema direto de condução de calor, atribuindo uma expressão
algébrica para a CTC em função do tempo e do ângulo (uma vez que a superfície de contato era circular), a fim de simular medidas de temperaturas
em determinadas posições. Em seguida, aplicaram o método do gradiente conjugado a fim de minimizar um funcional que envolvia valores calculados
de temperatura para uma estimativa de CTC e os valores medidos correspondentes, ao longo do tempo. Foram também realizadas simulações em que se
consideravam erros randômicos de distribuição gaussiana nas medições sintéticas. Os autores verificaram que, mesmo com a introdução de pequenos erros de medição e com
o aumento da distância dos pontos de medição em relação ao centro da interface circular, era possível obter estimativas confiáveis da CTC. O mesmo
problema foi formulado de forma não-linear e resolvido por \cite{artigo_huang_2}, gerando resultados ainda mais precisos.

\cite{artigo_milosevic} combinaram a técnica de estimação de parâmetros de Gauss com o método de pulso de \textit{laser} (originalmente empregado
na medição de difusividade térmica de materiais) a fim de estimar a CTC entre dois sólidos. Assim como no trabalho citado previamente, foram calculadas
medidas sintéticas de temperatura obtidas do problema direto, além de simulações envolvendo ruídos gaussianos adicionados a essas medidas.

A variação temporal da CTC também foi considerada por \cite{artigo_yang} ao estudar os efeitos presentes na interface entre um cabo de
fibra óptica e seu revestimento. De forma semelhante ao trabalho de \cite{artigo_huang}, foi aplicado o método do gradiente conjugado para estimação
da resistência térmica de contato nessa interface. Para a resolução do problema inverso, não foi necessário um conhecimento prévio da forma funcional das grandezas desconhecidas.

\cite{artigo_fieberg} relizaram investigações voltadas à análise da CTC de materiais aplicados em motores a combustão. Em seu trabalho, o salto de temperatura
na interface foi determinado através de medições obtidas com o uso de câmeras de infravermelho, e o fluxo de calor na interface foi calculado
através da resolução de um problema inverso de condução de calor; a razão entre essas grandezas forneceu a estimativa da CTC, de acordo
com a definição.   

Um problema inverso de condução de calor para estimativa de CTC entre corpos cujo contato varia ciclicamente foi analisado por \cite{artigo_shoj}.
Os autores levantaram duas classes de resultados, uma obtida por dados simulados com erros gaussianos, e outra com dados experimentais, obtidos através
de termopares poisicionados ao longo dos corpos de prova. O problema
inverso foi resolvido de forma iterativa para ambas as classes. Os valores de salto de temperatura na interface, necessários para o cálculo da CTC,
foram determinados de forma direta na primeira classe, e determinados através de regressão linear na segunda classe. 

A variação espacial da CTC ao longo da interface de contato foi considerada por \cite{artigo_gill}, que, na resolução do problema inverso, aplicaram
o método dos elementos de contorno \citep{livro_bem} associado a um algoritmo genético para regularização das medidas de temperatura. O procedimento empregado
envolvia a medição de temperaturas na interface. Uma simulação numérica do experimento foi implementada, sendo observada uma razoável sensibilidade
da solução em relação aos erros de medição. 

Duas características comums aos métodos até agora citados para determinação da CTC são: a) a necessidade de ter medidas de temperatura disponíveis na interface de contato,
normalmente por meios intrusivos, ou seja, medidas diretas de temperatura no interior dos corpos de prova; b) a necessidade de se fazer alguma descrição física e geométrica, em
escala microscópica, da interface de contato, o que pode exigir a separação dos materiais. Em face dessas dificuldades, \cite{reciproc_3} desenvolveram
um método não-intrusivo e não-destrutivo para estimativa da distribuição espacial da CTC em regime permanente. Para tanto, a fim de resolver o problema inverso de condução de calor, os autores partiram do conceito de funcional de reciprocidade, originalmente
empregado na identificação de falhas ou descontinuidades planas em materiais \citep{artigo_andrieux}.

A metodologia empregada por \cite{reciproc_3}, e que basicamente foi replicada nos trabalhos posteriores envolvendo funcionais de reciprocidade para estimativa
da CTC, consiste em duas etapas. Na primeira,
são formulados dois problemas difusivos auxiliares, no mesmo domínio do problema original; um problema é relacionado ao salto de temperatura na interface, e
o outro é relacionado ao fluxo de calor na interface. As soluções desses problemas são usadas para obter dois conjuntos de funções ortogonais; dessa forma,
o salto de temperatura e o fluxo de calor são expressos como combinações lineares das funções ortogonais correspondentes. A etapa seguinte é a determinação 
dos coeficientes dessas combinações lineares através dos funcionais de reciprocidade, que consistem em integrais calculadas nas fronteiras do domínio, envolvendo as soluções dos
problemas auxiliares resolvidos previamente, além de medidas externas de temperatura e fluxo de calor. As integrais são tais que
se anulam ao longo da interface de contato, sendo calculadas apenas nas fronteiras externas dos corpos de prova, sobre as quais conhecem-se as temperaturas
ou são impostas condições de contorno. Finalmente, obtidos os coeficientes das expansões lineares, são calculadas as estimativas de fluxo de calor e de salto de temperatura na interface;
a CTC é então avaliada através da razão entre essas quantidades. É importante destacar que a determinação da CTC por esta técnica é feita de forma não iterativa,
através da aplicação direta da definição. Outra vantagem desta técnica é que, para uma dada configuração geométrica dos corpos em contato, os problemas auxiliares que
fornecem as funções ortogonais só precisam ser resolvidos uma vez; desse modo, é possível estimar diferentes distribuições de CTC conhecendo-se apenas as medidas externas de temperatura
e de fluxo de calor, as quais, juntamente com as funções auxiliares, permitem calcular os funcionais de reciprocidade.

Os problemas auxiliares levantados por \cite{reciproc_3} são problemas de Cauchy, isto é, se caracterizam por terem duas condições de contorno sobre uma
mesma região da fronteira, o que requer o uso de técnicas especiais de solução. Os autores propuseram o método das soluções fundamentais \citep{artigo_marin}, aplicando-o
para seis diferentes perfis de CTC ao longo de uma interface de contato plana: um perfil constante ao longo da interface, dois perfis com variações suaves e três perfis apresentando descontinuidades. Para os
perfis sem descontinuidades (os três primeiros descritos acima), os resultados obtidos foram muito bons, mesmo após o acréscimo de erros gaussianos às medições
de temperatura. Para os casos de perfis descontínuos, os resultados permitiram recuperar o comportamento variacional das descontinuidades ao longo da interface.

\cite{artigo_colaco_2} estenderam o trabalho citado anteriormente, introduzindo um termo transiente no problema inverso de determinação da CTC, mantendo a mesma formulação
nos problemas auxiliares. A
metodologia adotada foi a mesma, incluindo o emprego da técnica das soluções fundamentais. Assim, chegaram a uma relação envolvendo os funcionais de reciprocidade,
que, no regime transiente, se reduzia ao resultado encontrado no trabalho anterior. Desse modo, a CTC encontrada era função do tempo, e sua estimativa era calculada 
iterativamente, melhorando à medida em que se aproximava do regime permanente.

Uma nova abordagem do tratamento transiente foi feita por \cite{artigo_colaco_4},
ao introduzir termos transientes nos problemas auxiliares, ainda aplicando a mesma metodologia baseada em funcionais de reciprocidade e soluções fundamentais. O tratamento
matemático envolvia a integração das funções de reciprocidade no tempo, eliminando o aspecto iterativo observado no trabalho anterior. Os resultados encontrados mostraram
as mesmas características dos trabalhos anteriores, indicando assim a robustez do método apresentado.

O método dos funcionais de reciprocidade foi empregado por \cite{tese_abreu} e \cite{artigo_abreu_3} na detecção indireta de falhas de contato planas em materiais bicompostos, através da estimativa da distribuição espacial da CTC.
Em situações práticas, tal falha é inacessível, e o método proposto se adequa por não ser intrusivo. Um aparato experimental, com descontinuidades na interface produzidas artificialmente, e a partir
do qual foram obtidas medidas reais de temperatura, foi construído para validar a modelagem. Para resolução dos problemas auxiliares, foi empregado o método 
de Monte Carlo com Cadeias de Markov \citep{artigo_mcmc}. A partir da estimativa do perfil da CTC, foi possível identificar e caracterizar qualitativamente as regiões de falha na interface,
de forma satisfatória.

\cite{artigo_padilha_3} avançaram no desenvolvimento do método dos funcionais de reciprocidade na estimativa da CTC ao aplicarem a reformulação das condições de contorno 
dos problemas auxiliares proposta por \cite{tese_abreu}, convertendo-os em problemas de valor de contorno e possibilitando o uso da Técnica da Transformada Integral Generalizada \citep{livro_integral_transforms_cotta}.
Uma definição conveniente das condições de contorno permitiu que o sistema linear que forneceria os coeficientes das expansões lineares se transformasse num sistema diagonal, simplificando
consideravelmente o cálculo desses coeficientes. Com isso, 
o problema inverso originalmente explorado por \cite{reciproc_3} foi desenvolvido analiticamente, fornecendo uma equação simples para o cálculo da CTC. O ganho
computacional obtido através desta abordagem foi considerável, uma vez que a distribuição da CTC pôde ser obtida em frações de segundo, de forma direta, obtendo resultados consistentes
com os encontrados através dos funcionais de reciprocidade empregando outras técnicas de solução.


