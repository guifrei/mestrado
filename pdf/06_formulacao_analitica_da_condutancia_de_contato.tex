\section{Formulação analítica para a condutância térmica de contato}

No capítulo anterior, foram finalmente deduzidas as expressões para as funções auxiliares $F_{1,j}$ e $G_{1,j}$, bem como as funções $\beta_j$ e $\gamma_j$ que compõem as bases ortonormais em $L^2(\Gamma)$\footnote{A fim de simplificar a notação, e manter consistência com as equações estabelecidas na seção \ref{secao_sobre_fr}, será eliminado o símbolo $\hat{}$, usado para indicar as funções ortonormalizadas, assumindo a partir deste ponto que $\beta_j$ e $\gamma_j$ referem-se às funções obtidas \textit{após} o processo de ortogonalização de Gram-Schmidt.}. Neste capítulo, estes resultados serão reunidos, a fim de levantar uma expressão analítica de uso geral, para estimativa da condutância térmica de contato ao longo da superfície irregular $\Gamma$. Como ponto de partida, será estabelecida a expressão em coordenadas cartesianas do produto interno entre as funções no espaço linear $L^2(\Gamma)$.

\subsection{Formulação do produto interno no espaço linear de funções $L^2(\Gamma)$ em coordenadas cartesianas}

Na seção \ref{secao_sobre_fr}, foi definido na equação \eqref{definicao_innner_product} o produto interno entre duas funções $f_1$ e $f_2$ no espaço linear de funções $L^2(\Gamma)$:
\begin{align}
\langle f_1, f_2\rangle = \int_\Gamma f_1(\Gamma) f_2(\Gamma) d\Gamma \label{integral_da_definicao_produto_interno}
\end{align}

Se a superfície $\Gamma$, sobre a qual é realizada a integração, possuir uma representação paramétrica da forma
\begin{align}
\Gamma(t) = x(t)\mathbf{a}_x + y(t)\mathbf{a}_y, \quad\quad\quad t_0 < t < t_1
\end{align}
onde $\mathbf{a}_x$ e $\mathbf{a}_y$ são os vetores unitários da base canônica do sistema cartesiano, então a integral \eqref{integral_da_definicao_produto_interno} pode ser escrita como \citep{livro_stewart_2}:
\begin{align}
\langle f_1, f_2\rangle = \int_{t=t_0}^{t=t_1}f_1(x(t), y(t))f_2(x(t), y(t))\sqrt{x'(t)^2 + y'(t)^2}dt \label{integral_da_definicao_produto_interno_2}
\end{align}

Adotando as parametrizações \eqref{parametrizacao_x} e \eqref{parametrizacao_y}, pode-se escrever:
\begin{align} 
\langle f_1, f_2\rangle = \int_{t=0}^{t=a}f_1(t, w(t))f_2(t, w(t))\sqrt{1 + w'(t)^2}dt \label{integral_da_definicao_produto_interno_3}
\end{align}
ou, uma vez que $t$ é uma variável muda,
\begin{align}
\langle f_1, f_2\rangle = \int_{x=0}^{x=a}f_1(x, w(x))f_2(x, w(x))\sqrt{1 + w'(x)^2}dx \label{integral_da_definicao_produto_interno_4}
\end{align}

\subsection{Formulação da estimativa da condutância térmica de contato em coordenadas cartesianas}\label{secao_com_funcionais}

Os funcionais de reciprocidade, conforme dito na seção \ref{secao_sobre_fr}, são ferramentas através das quais será estimada a condutância térmica de contato (CTC) na interface de contato $\Gamma$. A expressão \eqref{def_funcional_reciprocidade} fornece o funcional de reciprocidade para uma função $F(x, y)$:
\begin{align}
\Re(F)
=
\int_{\Gamma_0}\left[\left(\frac{-q}{k_1}\right)F - Y\frac{\partial F}{\partial\mathbf{n_1}}\right]d\Gamma_0
\label{integral_de_contorno_para_F}
\end{align}

A superfície $\Gamma_0$, representada no arranjo da figura \ref{fig2}, e sobre a qual é calculada a integral de contorno \eqref{integral_de_contorno_para_F}, pode ser parametrizada como:
\begin{align}
\Gamma_0(t) = t\mathbf{a}_x + b \mathbf{a}_y, \quad\quad\quad 0 < t < a
\end{align}
onde $b$ é a altura do corpo de prova representado na figura \ref{fig2}.

O vetor $\mathbf{n}_1$, normal à superfície $\Gamma_0$, é o próprio vetor unitário $\mathbf{a}_y$ da base canônica. Logo, a derivada direcional de $F$ sobre esse vetor será dada por \citep{livro_stewart_2}:
\begin{align}
\frac{\partial F}{\partial\mathbf{n}_1}\bigg|_{\Gamma_0} & = \nabla F \cdot \mathbf{a}_y \nonumber \\
& = \left[\frac{\partial F(x, b)}{\partial x}\mathbf{a}_x + \frac{\partial F(x, b)}{\partial y}\mathbf{a}_y \right] \cdot \mathbf{a}_y \nonumber \\
& = \frac{\partial F(x, b)}{\partial y}
\end{align} 

Substituindo na integral \eqref{integral_de_contorno_para_F}, obtém-se \citep{livro_stewart_2}:
\begin{align}
\Re(F)
=
\int_{t=0}^{t=a} \left[\left(\frac{-q}{k_1}\right)F(t, b) - Y(t)\frac{\partial F(t, b)}{\partial y}\right] dt
\label{integral_de_contorno_para_F_2}
\end{align}
ou
\begin{align}
\Re(F)
=
\int_{x=0}^{x=a} \left[\left(\frac{-q}{k_1}\right)F(x, b) - Y(x)\frac{\partial F(x, b)}{\partial y}\right] dx
\label{integral_de_contorno_para_F_3}
\end{align}

Dessa forma, aplicando a equação \eqref{integral_de_contorno_para_F_3} para expressar os funcionais de reciprocidade das funções $F_{1,j}$ e $G_{1,j}$:
\begin{align}
\Re(F_{1,j})
=
\int_{x=0}^{x=a} \left[\left(\frac{-q}{k_1}\right)F_{1,j}(x, b) - Y(x)\frac{\partial F_{1,j}(x, b)}{\partial y}\right] dx
\label{integral_de_contorno_para_F1_0}
\end{align}
\begin{align}
\Re(G_{1,j})
=
\int_{x=0}^{x=a} \left[\left(\frac{-q}{k_1}\right)G_{1,j}(x, b) - Y(x)\frac{\partial G_{1,j}(x, b)}{\partial y}\right] dx
\label{integral_de_contorno_para_G1_0}
\end{align}
ou, substituindo as condições de contorno \eqref{funcao_F_cc_T1_2_cart} e \eqref{funcao_G_cc_T1_2_cart}:
\begin{align}
\Re(F_{1,j})
& =
\int_0^a \left[\left(\frac{-q}{k_1}\right)\psi_j(x) - Y(x)\frac{\partial F_{1,j}(x, b)}{\partial y}\right] dx \nonumber \\
& =
-\frac{q}{k_1}\int_0^a\psi_j(x)dx - \int_0^a Y(x)\frac{\partial F_{1,j}(x, b)}{\partial y} dx \nonumber \\
& =
-\frac{q}{k_1}\bar{\psi}_{j,0} - \int_0^a Y(x)\frac{\partial F_{1,j}(x, b)}{\partial y} dx
\label{integral_de_contorno_para_F1}
\end{align}
%
\begin{align}
\Re(G_{1,j})
& =
\int_0^a \left[\left(\frac{-q}{k_1}\right)\phi_j(x) - Y(x)\frac{\partial G_{1,j}(x, b)}{\partial y}\right] dx \nonumber \\
& =
-\frac{q}{k_1}\int_0^a\phi_j(x)dx - \int_0^a Y(x)\frac{\partial G_{1,j}(x, b)}{\partial y} dx
\nonumber \\
& =
-\frac{q}{k_1}\bar{\phi}_{j,0} - \int_0^a Y(x)\frac{\partial G_{1,j}(x, b)}{\partial y} dx
\label{integral_de_contorno_para_G1}
\end{align}

As derivadas em relação a $y$ de $F_{1,j}$ e $G_{1,j}$, avaliadas em $y = b$, podem ser determinadas por derivação parcial das expressões \eqref{solucao_transf_inversa_F1_com_dependencia} e
\eqref{solucao_transf_inversa_G1_com_dependencia}, truncadas até o índice $M$, fornecendo:
%
\begin{align}
\frac{\partial F_{1, j}(x, b)}{\partial y} = & \frac{\bar{\psi}_{j,0} - \mathbb{A}_{j,0}}{ab} - 
\frac{2}{a}\sum_{m=1}^M \mu_m \left(\mathbb{A}_{j,m}\frac{1}{\sinh\mu_m b} - \bar{\psi}_{j, m}\frac{\cosh\mu_m b}{\sinh\mu_m b}\right)\cos\mu_m x \nonumber \\
= & \frac{\bar{\psi}_{j,0} - \mathbb{A}_{j,0}}{ab} - 
\frac{2}{a}\sum_{m=1}^M \mu_m \left(\frac{\mathbb{A}_{j,m}}{\sinh\mu_m b} - \frac{\bar{\psi}_{j, m}}{\tanh\mu_m b}\right)\cos\mu_m x
\label{derivada_parcial_y_F}
\end{align}
%
\begin{align}
\frac{\partial G_{1, j}(x, b)}{\partial y} = & \frac{\bar{\phi}_{j,0} - \mathbb{E}_{j,0}}{ab} - 
\frac{2}{a}\sum_{m=1}^M \mu_m \left(\mathbb{E}_{j,m}\frac{1}{\sinh\mu_m b} - \bar{\phi}_{j, m}\frac{\cosh\mu_m b}{\sinh\mu_m b}\right)\cos\mu_m x \nonumber \\
= & \frac{\bar{\phi}_{j,0} - \mathbb{E}_{j,0}}{ab} - 
\frac{2}{a}\sum_{m=1}^M \mu_m \left(\frac{\mathbb{E}_{j,m}}{\sinh\mu_m b} - \frac{\bar{\phi}_{j, m}}{\tanh\mu_m b}\right)\cos\mu_m x
\label{derivada_parcial_y_G}
\end{align}

Substituindo os resultados \eqref{derivada_parcial_y_F} e \eqref{derivada_parcial_y_G} em \eqref{integral_de_contorno_para_F1} e \eqref{integral_de_contorno_para_G1}, obtém-se:
\begin{align}
\Re(F_{1,j})
& =
-\frac{q}{k_1}\bar{\psi}_{j,0} + \frac{\mathbb{A}_{j,0} - \bar{\psi}_{j,0}}{ab}\int_0^a Y(x)dx + \nonumber \\
& \frac{2}{a}\sum_{m=1}^M \mu_m \left(\frac{\mathbb{A}_{j,m}}{\sinh\mu_m b} - \frac{\bar{\psi}_{j, m}}{\tanh\mu_m b}\right)\int_0^a Y(x)\cos\mu_m x dx
\label{calculo_FR_F1_antes}
\end{align}
%
\begin{align}
\Re(G_{1,j})
& =
-\frac{q}{k_1}\bar{\phi}_{j,0} + \frac{\mathbb{E}_{j,0} - \bar{\phi}_{j,0}}{ab}\int_0^a Y(x)dx + \nonumber \\
& \frac{2}{a}\sum_{m=1}^M \mu_m \left(\frac{\mathbb{E}_{j,m}}{\sinh\mu_m b} - \frac{\bar{\phi}_{j, m}}{\tanh\mu_m b}\right)\int_0^a Y(x)\cos\mu_m x dx
\label{calculo_FR_G1_antes}
\end{align}

As equações \eqref{calculo_FR_F1_antes} e \eqref{calculo_FR_G1_antes} fornecem os funcionais de reciprocidade $\Re(F_{1,j})$ e $\Re(G_{1,j})$ referentes à uma escolha particular de funções auxiliares $\psi_j(x)$ e $\phi_j(x)$, respectivamente. Os termos $\bar{\psi}_{j, m}$ e $\bar{\phi}_{j, m}$, por sua vez, correspondem respectivamente às transformadas integrais das referidas funções $\psi_j(x)$ e $\phi_j(x)$. 

Finalmente, conhecidos os funcionais de reciprocidade, bem como as funções de base ortogonal $\beta_j(x)$ e $\gamma_j(x)$, aplica-se a equação \eqref{equacao_definicao_f_r}, deduzida na seção \ref{secao_sobre_fr}:
\begin{align}
& h_c(x) % = \frac{- k_1 \displaystyle\frac{\partial T_1}{\partial\mathbf{n_1}}\bigg|_\Gamma}{[T_1 - T_2]_\Gamma} 
= \frac{\displaystyle\sum_{j=1}^{N_2} \Re(G_{1,j}) \gamma_j(x)}{\displaystyle\sum_{j=1}^{N_1} \Re(F_{1,j}) \beta_j(x)} \label{expressao_final_ctc}
\end{align}

A expressão \eqref{expressao_final_ctc} e as relações \eqref{calculo_FR_F1_antes} e \eqref{calculo_FR_G1_antes} representam uma generalização da expressão analítica para o cálculo da condutância térmica de contato obtida por \cite{tese_padilha} para o caso em que a interface de contato $\Gamma$ é plana e paralela às bases do corpo de prova. Assim como o resultado encontrado para o referido caso, elas permitem estimar de forma direta a distribuição espacial da CTC ao longo da interface $\Gamma$, através das medidas de temperatura $Y(x)$ tomadas sobre a superfície $\Gamma_0$, conforme havia sido comentado na seção \ref{secao_sobre_fr}. Essas expressões envolvem integrais da forma $\displaystyle \int_0^a Y(x)dx$ e $\displaystyle \int_0^a Y(x)\cos\mu_m x dx$, também presentes no trabalho de \cite{tese_padilha}, que podem ser calculadas previamente e aplicadas nos somatórios \eqref{calculo_FR_F1_antes} e \eqref{calculo_FR_G1_antes}. A determinação dos coeficientes $\mathbb{A}_{j,m}$ e $\mathbb{E}_{j,m}$ foi discutida nas seções anteriores.


%%%%%TODO
%\newpage
%\begin{align}
%\beta_j(x) = \frac{k_1}{\sqrt{1 + w'(x)^2}}\sum_{m=0}^M [\mathbb{A}_{j,m}\xi_m(x) - \bar{\psi}_{j, m}\omega_m(x)]
%\end{align}
%
%\begin{align}
%\xi_m(x) = \left\lbrace
%\begin{array}{ll}
%\displaystyle \frac{1}{ab}, & m = 0 \\ \nonumber \\
%\displaystyle \frac{2\mu_m[\cos\mu_m x\cosh\mu_m v(x) - w(x)\sin\mu_m x\sinh\mu_m v(x)]}{a\sinh\mu_m b}, & m \ne 0
%\end{array}
%\right.
%\end{align}
%
%\begin{align}
%\omega_m(x) = \left\lbrace
%\begin{array}{ll}
%\displaystyle \frac{1}{ab}, & m = 0 \\ \nonumber \\
%\displaystyle \frac{2\mu_m[w'(x)\sin\mu_m x\sinh\mu_m w(x) + \cos\mu_m x\cosh\mu_m w(x)]}{a\sinh\mu_m b}, & m \ne 0
%\end{array}
%\right.
%\end{align}
%
%\begin{align}
%\gamma_j(x) = \sum_{m=0}^M [\mathbb{E}_{j,m}\chi_m(x) + \bar{\phi}_{j, m}\lambda_m(x)]
%\end{align} 
%
%\begin{align}
%\chi_m(x) = \left\lbrace
%\begin{array}{ll}
%\displaystyle \frac{b - w(x)}{ab}, & m = 0 \\ \nonumber \\
%\displaystyle \frac{2\sinh\mu_m [b - w(x)]}{a\sinh\mu_m b}, & m \ne 0
%\end{array}
%\right.
%\end{align}
%
%\begin{align}
%\lambda_m(x) = \left\lbrace
%\begin{array}{ll}
%\displaystyle \frac{w(x)}{ab}, & m = 0 \\ \nonumber \\
%\displaystyle \frac{2\sinh\mu_m w(x)}{a\sinh\mu_m b}, & m \ne 0
%\end{array}
%\right.
%\end{align}
%
%\begin{align}
%[T_1 - T_2]_\Gamma & = \sum_{j=1}^{N_1} k_1 \Re(F_{1,j}) \sum_{m=0}^M [\mathbb{A}_{j,m}\xi_m(x) - \bar{\psi}_{j, m}\omega_m(x)] \nonumber \\
%& = k_1 \sum_{m=0}^M \sum_{j=1}^{N_1} \Re(F_{1,j}) [\mathbb{A}_{j,m}\xi_m(x) - \bar{\psi}_{j, m}\omega_m(x)] \nonumber \\
%\end{align}
%
%\begin{align}
%\langle \beta_j, \beta_k\rangle = k_1^2 \sum_{m=0}^M \sum_{n=0}^M \int_0^a \frac{[\mathbb{A}_{j,m}\xi_m(x) - \bar{\psi}_{j, m}\omega_m(x)][\mathbb{A}_{k,n}\xi_n(x) - \bar{\psi}_{k, n}\omega_n(x)]}{\sqrt{1 + w'(x)^2}} dx
%\end{align}
%
%\newpage
%
%%%%%%%%%%%%%TODO
%\begin{align}
%Y(x) \approx \sum_{n=0}^M y_n \cos \mu_n(x)
%\end{align}
%
%\begin{align}
%& \int_0^a Y(x) dx = y_0 a \nonumber \\
%& \int_0^a Y(x) \cos\mu_m x dx = \frac{y_m a}{2}
%\end{align}
%
%\begin{align}
%\Re(F_{1,j})
%& =
%-\frac{q}{k_1}\bar{\psi}_{j,0} + \frac{\mathbb{A}_{j,0} - \bar{\psi}_{j,0}}{b}y_0 + 
%\sum_{m=1}^M \mu_m \left(\frac{\mathbb{A}_{j,m}}{\sinh\mu_m b} - \frac{\bar{\psi}_{j, m}}{\tanh\mu_m b}\right)y_m
%\end{align}
%%
%\begin{align}
%\Re(G_{1,j})
%& =
%-\frac{q}{k_1}\bar{\phi}_{j,0} + \frac{\mathbb{E}_{j,0} - \bar{\phi}_{j,0}}{b}y_0 + 
%\sum_{m=1}^M \mu_m \left(\frac{\mathbb{E}_{j,m}}{\sinh\mu_m b} - \frac{\bar{\phi}_{j, m}}{\tanh\mu_m b}\right)y_m
%\end{align}
%
%\begin{align}
%& \frac{\displaystyle \mu_{m+1} \left(\frac{\mathbb{A}_{j,{m+1}}}{\sinh\mu_{m+1} b} - \frac{\bar{\psi}_{j, {m+1}}}{\tanh\mu_{m+1} b}\right)y_{m+1}}{\displaystyle \mu_m \left(\frac{\mathbb{A}_{j,m}}{\sinh\mu_m b} - \frac{\bar{\psi}_{j, m}}{\tanh\mu_m b}\right)y_m} = \nonumber \\
%& \frac{\mu_{m+1}}{\mu_m} \frac{y_{m+1}}{y_m} \frac{\displaystyle  \frac{\mathbb{A}_{j,{m+1}}}{\sinh\mu_{m+1} b} - \frac{\bar{\psi}_{j, {m+1}}}{\tanh\mu_{m+1} b}}{\displaystyle \frac{\mathbb{A}_{j,m}}{\sinh\mu_m b} - \frac{\bar{\psi}_{j, m}}{\tanh\mu_m b}} = \nonumber \\
%& \frac{\mu_{m+1}}{\mu_m} \frac{y_{m+1}}{y_m} \frac{\displaystyle \frac{2\mathbb{A}_{j,m+1}}{e^{\mu_{m+1} b} - e^{-\mu_{m+1} b}} - \frac{\bar{\psi}_{j, m+1}(e^{\mu_{m+1} b} + e^{-\mu_{m+1} b})}{e^{\mu_{m+1} b} - e^{-\mu_{m+1} b}}}{\displaystyle \frac{2\mathbb{A}_{j,m}}{e^{\mu_m b} - e^{-\mu_m b}} - \frac{\bar{\psi}_{j, m}(e^{\mu_m b} + e^{-\mu_m b})}{e^{\mu_m b} - e^{-\mu_m b}}} = \nonumber \\
%%
%& \frac{\mu_{m+1}}{\mu_m} \frac{y_{m+1}}{y_m} \frac{\displaystyle \frac{2\mathbb{A}_{j,m+1}e^{-\mu_{m+1} b}}{1 - e^{-2\mu_{m+1} b}} - \frac{\bar{\psi}_{j, m+1}(1 + e^{-2\mu_{m+1} b})}{1 - e^{-2\mu_{m+1} b}}}{\displaystyle \frac{2\mathbb{A}_{j,m}e^{-\mu_m b}}{1 - e^{-2\mu_m b}} - \frac{\bar{\psi}_{j, m}(1 + e^{-2\mu_m b})}{1 - e^{-2\mu_m b}}} = \nonumber \\
%%
%& \frac{\mu_{m+1}}{\mu_m} \frac{y_{m+1}}{y_m} \frac{1 - e^{-2\mu_m b}}{1 - e^{-2\mu_{m+1} b}}\frac{2\mathbb{A}_{j,m+1}e^{-\mu_{m+1} b} - \bar{\psi}_{j, m+1}(1 + e^{-2\mu_{m+1} b})}{2\mathbb{A}_{j,m}e^{-\mu_m b} - \bar{\psi}_{j, m}(1 + e^{-2\mu_m b})} \approx \nonumber \\
%& \frac{m+1}{m} \frac{y_{m+1}}{y_m} \frac{2\mathbb{A}_{j,m+1}e^{-\mu_{m+1} b} - \bar{\psi}_{j, m+1}}{2\mathbb{A}_{j,m}e^{-\mu_m b} - \bar{\psi}_{j, m}}
%\end{align}
%
%\begin{align}
%& \int_0^a \tilde{Y}(x) \cos\mu_m x dx = \int_0^a Y(x) \cos\mu_m x dx + \sigma\int_0^a \epsilon(x) \cos\mu_m x dx \nonumber \\
%& \therefore \frac{\tilde{y}_m a}{2} = \frac{y_m a}{2} + \sigma\int_0^a \epsilon(x) \cos\mu_m x dx\nonumber \\
%& \therefore \frac{\tilde{y}_m a}{2\sigma} = \frac{y_m a}{2\sigma} + \int_0^a \epsilon(x) \cos\mu_m x dx\nonumber \\
%& \therefore \abs{\frac{\tilde{y}_m a}{2\sigma}} \le \abs{\frac{y_m a}{2\sigma}} + \abs{\int_0^a \epsilon(x) \cos\mu_m x dx} \nonumber \\
%& \Rightarrow \abs{\frac{\tilde{y}_m a}{2\sigma}} - \abs{\frac{y_m a}{2\sigma}} \le \abs{\int_0^a \epsilon(x) \cos\mu_m x dx}
%\end{align}
%
%Cauchy-Schwartz:
%\begin{align}
%\left[\int_0^a \epsilon(x) \cos\mu_m x dx\right]^2 \le \int_0^a \epsilon(x)^2 dx \int_0^a(\cos\mu_m x)^2 dx
%\end{align}
%
%Mas
%\begin{align}
%\int_0^a(\cos\mu_m x)^2 dx = \frac{a}{2}
%\end{align}
%
%Logo:
%\begin{align}
%& \left[\int_0^a \epsilon(x) \cos\mu_m x dx\right]^2 \le \frac{a}{2}\int_0^a \epsilon(x)^2 dx \nonumber \\
%& \Rightarrow \abs{\int_0^a \epsilon(x) \cos\mu_m x dx} \le \sqrt{\frac{a}{2}}\sqrt{\int_0^a \epsilon(x)^2 dx}
%\end{align}
%
%Assim:
%\begin{align}
%& \abs{\frac{\tilde{y}_m a}{2\sigma}} - \abs{\frac{y_m a}{2\sigma}} \le \sqrt{\frac{a}{2}}\sqrt{\int_0^a \epsilon(x)^2 dx} \nonumber \\
%& \therefore \abs{\tilde{y}_m} - \abs{y_m} \le \sigma\sqrt{\frac{2}{a}}\sqrt{\int_0^a \epsilon(x)^2 dx}
%\end{align}
%
%Mas (Qui-quadrado):
%\begin{align}
%\frac{1}{a}\int_0^a \epsilon(x)^2 dx \approx 1 \Rightarrow \int_0^a \epsilon(x)^2 dx \approx a
%\end{align}
%
%Logo:
%\begin{align}
%\abs{\tilde{y}_m} - \abs{y_m} \le \sigma\sqrt{2a}
%\end{align}
%
%\begin{align}
%\frac{1}{\sinh z} = \frac{2}{e^z - e^{-z}} = \frac{2e^{-z}}{1 - e^{-2z}} \approx 2e^{-z}
%\end{align}
%
%\begin{align}
%\frac{1}{\tanh z} = \frac{e^z + e^{-z}}{e^z - e^{-z}} = \frac{1 + e^{-2z}}{1 - e^{-2z}} \approx 1
%\end{align}
%
%\begin{align}
%\frac{\sinh z_1}{\sinh z_2} = \frac{e^{z_1} - e^{-z_1}}{e^{z_2} - e^{-z_2}}
% = \frac{e^{z_1-z_2} - e^{-z_1-z_2}}{1 - e^{-2z_2}}
%  = \frac{e^{-(z_2-z_1)} - e^{-(z_2+z_1)}}{1 - e^{-2z_2}}
%  \approx e^{-(z_2-z_1)}
%\end{align}
%
%Aproximação por mínimos quadrados:
%\begin{align}
%Y(x_k) \approx \sum_{n=0}^M y_n \cos \mu_n(x_k), k=0,1,2,..., K
%\end{align}
%
%Ou
%\begin{align}
%\mathbf{A}\mathbf{c} \approx \mathbf{y}
%\end{align}

%Na próxima seção, será feita uma validação dos resultados até então encontrados, aplicando-os para o caso estudado por \cite{tese_padilha}, a fim de deduzir a mesma expressão analítica de estimativa de CTC encontrada naquele trabalho.
%
%\subsection{Validação para o caso da interface plana}
%
%Será feita agora a análise dos resultados obtidos para o caso resolvido analiticamente por \cite{tese_padilha}, e que motivou o esenvolvimento do presente trabalho. O arranjo equivalente foi representado na figura \ref{fig4}, reproduzida a seguir.
%\begin{figure}[h!b]
%	\begin{center}
%		\begin{tikzpicture}[scale=0.7]
%		
%		\draw [ultra thick] (0, 0) -- (10, 0);
%		%\draw [ultra thick] (0, 2) -- (7, 2);
%		%\draw [ultra thick] (8, 2) -- (10, 2);
%		\draw [ultra thick] (0, 5) -- (10, 5);
%		\draw [ultra thick] (0, 2.0) -- (3.0, 2.0);
%		\draw [ultra thick] (3.6, 2.0) -- (10, 2.0);
%		\draw [ultra thick] (0, 0) -- (0, 5);
%		\draw [ultra thick] (10, 0) -- (10, 5);
%		
%		\draw (2.5, 3.5) node {$\Omega_1$};
%		\draw (7.5, 1) node {$\Omega_2$};
%		\draw (3.3, 2.2) node {$\Gamma$};
%		\draw (-0.5, 3.5) node {$\Gamma_1$};
%		\draw (-0.5, 1) node {$\Gamma_2$};
%		\draw (10.5, 3.5) node {$\Gamma_1$};
%		\draw (10.5, 1) node {$\Gamma_2$};
%		\draw (5, -0.4) node {$\Gamma_\infty$};
%		\draw (2.5, 5.3) node {$\Gamma_0$};
%		\draw [blue](5, 6.5) node {$q$};
%		\draw (5, -1.3) node {$a$};
%		\draw (-3.9, 2.5) node {$b$};
%		\draw (-2.5, 1) node {$b_2$};
%		\draw (-2.5, 3.5) node {$b_1$};
%		
%		\node [above right] at (12, 0) {$x$};
%		\node [right] at (0, 8) {$y$};
%		
%		\draw [->, blue] (0, 6.2) -- (0, 5.2);
%		\draw [->, blue] (1, 6.2) -- (1, 5.2);
%		\draw [->, blue] (2, 6.2) -- (2, 5.2);
%		\draw [->, blue] (3, 6.2) -- (3, 5.2);
%		\draw [->, blue] (4, 6.2) -- (4, 5.2);
%		\draw [->, blue] (5, 6.2) -- (5, 5.2);
%		\draw [->, blue] (6, 6.2) -- (6, 5.2);
%		\draw [->, blue] (7, 6.2) -- (7, 5.2);
%		\draw [->, blue] (8, 6.2) -- (8, 5.2);
%		\draw [->, blue] (9, 6.2) -- (9, 5.2);
%		\draw [->, blue] (10, 6.2) -- (10, 5.2);
%		
%		\draw [->, very thick] (10.2,0) -- (12,0);
%		\draw [->, very thick] (0, 6.5) -- (0,8);
%		
%		\draw [-] (0, -0.2) -- (0, -1.2);
%		\draw [-] (10, -0.2) -- (10, -1.2);
%		\draw [<->] (0, -1) -- (10, -1);
%		
%		\draw [-] (-0.2, 0) -- (-3.8, 0);
%		\draw [-] (-0.2, 5) -- (-3.8, 5);
%		\draw [-] (-0.2, 2) -- (-2.2, 2);
%		\draw [<->] (-3.6, 0) -- (-3.6, 5);
%		\draw [<->] (-2.0, 0) -- (-2.0, 2);
%		\draw [<->] (-2.0, 2) -- (-2.0, 5);
%		
%		\end{tikzpicture}
%		\caption{Geometria do problema físico: interface plana}
%		\label{fig8}
%	\end{center}
%\end{figure}
%
%Para o arranjo em questão, a curva que representa a interface de contato $\Gamma$ é simplesmente:
%\begin{align}
%w(x) = b_2 \label{curva_w_interface_plana}
%\end{align}
%
%Consequentemente,
%\begin{align}
%b - w(x) = b_1 \label{curva_w_interface_plana_2}
%\end{align}
%
%A constatação imediata é que, uma vez que $w'(x) = 0$, várias integrais envolvendo este termo encontradas ao longo da seção \ref{secao_probs_aux} serão anuladas, enquanto outras terão uma expressão mais simples. Além disso, \cite{tese_padilha} adota as seguintes funções auxiliares para definição das condições de contorno \eqref{funcao_F_cc_T1_2_cart} e \eqref{funcao_G_cc_T1_2_cart}:
%\begin{align}
%\psi_j(x) = \phi_j(x) = \left\lbrace
%	\begin{array}{ll}
%		\displaystyle\sqrt{\frac{1}{a}}, & j = 0\\ \\
%		\displaystyle\sqrt{\frac{2}{a}}\cos\mu_j x, & j \ne 0
%	\end{array}
%\right .
%\end{align}
%
%\begin{align}
%& p_m(x) = \left\lbrace
%\begin{array}{ll}
%\displaystyle -\frac{k_1}{b}, & m = 0 \\ \\
%\displaystyle -2k_1 \mu_m\frac{\cosh\mu_m b_1}{\sinh\mu_m b}X_m(x), & m \ne 0
%\end{array}
%\right. \label{compacta_p_cons} \\
%& q_m(x) = \left\lbrace
%\begin{array}{ll}
%\displaystyle -\frac{k_2}{b}, & m = 0 \\ \\
%\displaystyle - 2k_2 \mu_m\frac{\cosh\mu_m b_2}{\sinh\mu_m b} X_m(x), & m \ne 0
%\end{array}
%\right. \label{compacta_q_cons} \\
%&
%r_{j}(x) = -\frac{k_1}{b}\psi_{j,0} -  2k_1 \sum_{m=1}^M \bar{\psi}_{j, m} \mu_m\frac{\cosh\mu_m b_2}{\sinh\mu_m b}X_m(x) \label{compacta_r_cons}\\
%& u_m(x) = \left\lbrace
%\begin{array}{ll}
%\displaystyle -\frac{1}{b}, & m = 0 \\ \\
%\displaystyle -2 \mu_m\frac{\cosh\mu_m b_1}{\sinh\mu_m b}X_m(x), & m \ne 0
%\end{array}
%\right.  \label{compacta_p2_cons} \\
%&
%v_{j}(x) = -\frac{1}{b}\phi_{j,0} - 2 \sum_{m=1}^M \bar{\phi}_{j, m}\mu_m\frac{\cosh\mu_m b_2}{\sinh\mu_m b}X_m(x) \label{compacta_r2_cons}
%\end{align}
%
%\begin{align}
%\bar{v}_{j, n} & = -\frac{1}{b}\phi_{j,0} \int_0^a X_n(x) dx - 2 \sum_{m=1}^M \bar{\phi}_{j, m}\mu_m\frac{\cosh\mu_m b_2}{\sinh\mu_m b}\int_0^a X_m(x)X_n(x) dx
%\end{align}
%
%Para $j = 0$:
%\begin{align}
%\bar{v}_{0, n} & = -\frac{1}{b}\phi_{0,0} \int_0^a X_n(x) dx - 2 \sum_{m=1}^M \bar{\phi}_{0, m}\mu_m\frac{\cosh\mu_m b_2}{\sinh\mu_m b}\int_0^a X_m(x)X_n(x) dx
%\end{align}
%
%Mas $\bar{\phi}_{0, m} = 0$ e $\phi_{0,0} =\displaystyle  \int_0^a \sqrt{\frac{1}{a}} X_0(x)dx =  \sqrt{a}$, logo:
%\begin{align}
%\bar{v}_{0, n} & = -\frac{\sqrt{a}}{b} \int_0^a X_n(x) dx
%\end{align}
%
%Para $n = 0$:
%\begin{align}
%\bar{v}_{0, 0} = -\frac{a\sqrt{a}}{b}
%\end{align}
%
%Para $n \ne 0$:
%\begin{align}
%\bar{v}_{0, n} & = 0
%\end{align}
%
%Para $j \ne 0$:
%\begin{align}
%\bar{v}_{j, n} & = -\frac{1}{b}\phi_{j,0} \int_0^a X_n(x) dx - 2 \sum_{m=1}^M \bar{\phi}_{j, m}\mu_m\frac{\cosh\mu_m b_2}{\sinh\mu_m b}\int_0^a X_m(x)X_n(x) dx
%\end{align}
%
%Para $n = 0$:
%\begin{align}
%\bar{v}_{j, 0} = 0
%\end{align}
%
%Para $n \ne 0, n \ne j$:
%\begin{align}
%\bar{v}_{j, n} = 0
%\end{align}
%
%Para $n \ne 0, n  = j$:
%\begin{align}
%\bar{v}_{j, n} = - \sqrt{\frac{a}{2}} \mu_n\frac{\cosh\mu_n b_2}{\sinh\mu_n b}
%\end{align}