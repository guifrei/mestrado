\section{Revisão bibliográfica}

A revisão bibliográfica exposta nesta seção está dividida em duas partes. Na primeira parte, é feito um levantamento sobre a evolução histórica
das técnicas de solução de problemas difusivos, uma vez que, conforme será visto nos próximos capítulos, o problema inverso de condução de calor
a partir do qual será estimada a CTC é um problema difusivo. Na segunda parte, é apresentado um apanhado dos trabalhos clássicos que envolvem o tema da
determinação da CTC, antes e depois da introdução da abordagem deste tema como um problema inverso.

\subsection{Técnicas de soluções de problemas difusivos}

Segundo \cite{livro_tanehill}, há basicamente três abordagens ou métodos que podem ser usados para resolver um problema
de mecânica dos fluidos e/ou transferência de calor: experimental, teórico (ou analítico) e computacional (ou numérico). O primeiro
fornece resultados mais realistas a um custo de implementação maior. Já o segundo faz suposições
simplificadoras a fim de facilitar o tratamento do problema, e possivelmente encontrar uma solução fechada, ou seja, que é expressa geralmente como uma fórmula
matemática; uma vez que não envolve iterações ou interpolações, a obtenção do valor da solução num determinado ponto do domínio é praticamente imediata,
com baixo custo computacional. No último método -- a abordagem numérica -- as equações que governam o fenômeno em estudo são substituídas por
esquemas numéricos, cuja solução (obtida através de cálculos manuais ou pelo emprego de computadores digitais) é usada para representar de forma
aproximada a solução do problema original.

A teoria matemática moderna sobre a condução de calor foi estabelecida por Joseph Fourier, que reuniu suas investigações teóricas em sua obra
\textit{Théorie analytique de la chaleur} \citep{livro_fourier, artigo_langer}. Neste trabalho, Fourier deduz analiticamente as equações de condução para vários
tipos de sólidos, como esferas, prismas retangulares ou cilindros de seção reta circular. Nesses problemas, o fluxo de calor tinha
sempre uma direção que favorecia o aspecto simétrico do sólido em questão; por exemplo, no caso do cilindro ou da esfera, o fluxo
de calor acontecia na direção radial. Em seguida, Fourier apresenta a formulação tridimensional em coordenadas cartesianas da equação
de condução de calor em regime transiente, e a emprega para levantar as equações obtidas previamente, através de substituições de
variáveis. Finalmente, Fourier desenvolve o conceito de representação de uma função como um somatório infinito de funções
trigonométricas\footnote{A história registra que, quando Fourier apresentou seus artigos sobre a representação de funções arbitrárias
como expansões de senos e cossenos à Academia de Ciências de Paris em 1807 e 1811, recebeu críticas dos consultores (principalmente Lagrange,
que negou veementemente essa possibilidade), devido à falta de rigor, e por isso os artigos não foram publicados \citep{livro_agarwal}.} e, através da aplicação
da técnica de separação de variáveis, usa esse conceito na resolução analítica dos problemas que propôs no início do trabalho. Uma vez que os domínios envolvidos e as
condições de contorno impostas nos problemas possuíam algum tipo de simetria, que simplificava a variação da temperatura para apenas uma
variável dimensional (distância radial, comprimento ao longo de um eixo, etc.), as soluções encontradas eram relativamente simples, o que não
diminui sua importância por modelarem matematicamente o fenômeno físico da condução de calor pela primeira vez. O tratamento
dado por Fourier aos problemas de condução de calor foi, em suma, predominantemente analítico, sem considerações de caráter numérico.

Historicamente, os livros acadêmicos que versam sobre transferência de calor reproduzem a dedução da equação de condução em coordenadas
cartesianas feita por Fourier e em seguida apresentam a formulação equivalente nos sistemas de coordenadas cilíndricas e esféricas; ver
por exemplo \cite{livro_carslaw}, \cite{livro_holman} e \cite{livro_ozisik}. De fato, a formulação no sistema cartesiano é aplicável
a qualquer tipo de geometria; a solução geral sempre poderá ser expressa em termos de somatórios de senos ou cossenos. Soluções analíticas obtidas
nos sistemas de coordenadas esféricas ou cilíndricas envolvem somatórios de tipos diferentes de funções base ortogonais, obtidas através da aplicação do método
de separação de variáveis \citep{livro_boyce}. Voltando a citar os trabalhos de Fourier, ao expressar
as equações de condução para corpos esféricos ou cilíndricos, ele fez implicitamente um \textit{mapeamento} dessas geometrias a partir
do sistema de coordenadas cartesiano para os sistemas de coordenadas cilíndricas ou esféricas \citep{livro_numerical_grid}, e, ao aplicar a
separação de variáveis, encontra soluções em forma de séries que futuramente seriam associadas às funções de Bessel e Legendre \citep{livro_fourier}. 

\cite{artigo_einsenhart, artigo_einsenhart_2} demonstrou que a equação de Helmholtz (da qual a equação
de difusão é um caso particular) pode ser resolvida por separação de variáveis em onze diferentes
sistemas de coordenadas ortogonais, dentre as quais figuram os sistemas cartesiano, esférico e cilíndrico, bem como outros menos usuais, tais
como elíptico-cilíndrico ou parabólico. \cite{livro_moon} fizeram um extenso levantamento das equações diferenciais ordinárias
oriundas da separação de variáveis em cada um desses sistemas, e apresentam suas respectivas soluções gerais (dentre as quais estão as funções
seno e cosseno e as funções de Bessel e Legendre, já citadas). Tais equações, juntamente com as respectivas condições de contorno, constituem-se
em \textit{problemas de valor de contorno de Sturm-Liouville} \citep{artigo_sturm, artigo_liouville}. Problemas de Sturm-Liouville são na realidade
parte de uma teoria fundamental da Álgebra Linear conhecida como teoria dos operadores lineares; isto confere características e propriedades
especiais às soluções desses problemas (denominadas \textit{autofunções}), tais como a ortogonalidade e a possibilidade de, sob certas condições, expressar uma função arbitrária
como uma combinação linear dessas autofunções \citep{livro_boyce, livro_axler}.

Os métodos numéricos de solução de equações diferenciais ganharam grande impulso a partir da década de 1960, quando houve uma maior disponibilidade
de computadores digitais de alto desempenho \citep{livro_tanehill}. Aliado a esse fato, houve também um aumento natural da complexidade dos problemas de transferência
de calor e mecânica dos fluidos: adoção de geometrias complexas ou não convencionais, formulações de condições de contorno que introduziam dificuldades na
aplicação dos métodos analíticos, menores restrições quanto a não linearidade dos problemas. Já as abordagens numéricas, contudo, remontam a épocas anteriores. 

Uma dessas abordagens, considerada pioneira na análise numérica de equações diferenciais parciais em problemas
difusivos, foi proposta por \cite{artigo_richardson}. Em seu artigo, Richarsdon resolve numericamente a equação de Laplace e a equação biarmônica, e aplica esta última no problema prático de estudo da distribuição de tensão numa
barragem de alvenaria, com uma geometria bidimensional. Para tanto, ele representa o perfil da barragem usando segmentos de reta e emprega uma malha estruturada regular em seu interior. A malha
era resolvida usando um esquema iterativo de diferenças finitas centradas \citep{livro_tanehill}. Richardson já destacava as limitações dos métodos analíticos na integração de equações diferenciais parciais nos casos em que as fronteiras têm formato irregular.

Os anos seguintes testemunharam uma grande quantidade de pesquisas em mé-todos numéricos, especialmente no campo dos problemas de dinâmica dos fluidos. Avanços na área dos
problemas difusivos envolvem técnicas de precondicionamento de sistemas lineares e o desenvolvimento de esquemas de sobrerrelaxação, aumentando significativamente
o desempenho da taxa de convergência da solução \citep{artigo_frankel, artigo_fedorenko}.

Em meados da década de 1970, sistemas de coordenadas generalizadas, coincidentes com as fronteiras de domínios irregulares,
começam a ser empregados em detrimento dos sistemas de coordenadas ortogonais convencionais na resolução numérica de problemas advectivos-difusivos,
com o objetivo de evitar interpolações entre pontos da malha não coincidentes com as fronteiras \citep{livro_maliska}. As fronteiras do domínio físico, definidas geralmente no sistema 
cartesiano, são mapeadas para uma região retangular no domínio computacional, definido no sistema de coordenadas generalizado, através de relações de transformação.

A distribuição dos pontos na malha estruturada do problema físico original é feita de modo que a malha correspondente no domínio computacional seja regular. 
As equações diferenciais e as condições de contorno devem ser reescritas nesse novo sistema; as equações resultantes são mais complicadas uma vez que contêm mais
termos e coeficientes variáveis. Tais equações podem ser discretizadas e resolvidas usando, por exemplo, a técnica das diferenças finitas aplicada à malha retangular definida no domínio computacional.
Finalmente, a solução é transformada de volta para o domínio físico através das relações de transformação. Destaca-se o trabalho de \citet{artigo_thompson},
que desenvolveram uma técnica para determinação da relação de transformação entre sistemas de coordenadas para o caso bidimensional através da resolução numérica de equações diferenciais parciais.

Outra técnica de transformação de coordenadas envolve o emprego de transformações conformes \citep{livro_numerical_grid}. Uma grande vantagem
dessa técnica é o fato de que a equação de difusão preserva a forma original, sem introdução de termos não-lineares ou derivadas cruzadas;
por outro lado é aplicável apenas a geometrias bidimensionais. Se as fronteiras da geometria do problema não tiverem uma descrição algébrica
conhecida, também deve se fazer uso de esquemas de geração numérica de malha, baseados por exemplo na transformação de Schwarz-Christoffel \citep{livro_brown}.

Voltando ao trabalho de Fourier, é possível identificar em seu trabalho uma das bases sobre a qual se estabeleceu a técnica analítica de solução
de equações diferenciais parciais conhecida como transformação integral. No tratado previamente citado \textit{Théorie analytique de la chaleur},
procurando estender suas ideias para funções definidas em intervalos infinitos, Fourier descobriu uma fórmula de transformação integral e
sua respectiva inversa, que hoje levam o seu nome \citep{livro_integral_transforms}.

A técnica da transformação integral revelou-se uma
ferramenta poderosa para resolver problemas de valor inicial e problemas de valor de contorno para equações diferenciais parciais lineares.
O conceito de transformação integral originou-se dos trabalhos de Laplace, que publicou os primeiros resultados envolvendo a sua transformada
no tratado de 1812, \textit{Théorie analytique des probabilités} \citep{livro_integral_transforms}. Cauchy publicou em 1843 uma descrição
dos métodos simbólicos ou operacionais (que trabalham com operadores diferenciais como se fossem algébricos) \citep{livro_cauchy} e sua relação com a transformada de
Laplace, além de apresentar a forma exponencial da transformada de Fourier. Contudo, quem popularizou o uso das transformadas de Laplace foi Oliver Heaviside,
aplicando-as na resolução de equações diferenciais lineares em problemas de circuitos elétricos, especialmente a equação do telégrafo, consolidando as bases do
cálculo operacional moderno \citep{livro_yavetz, artigo_carson}. 

A grande vantagem oferecida pela transformação integral aplicada às equações de difusão linear é a possibilidade de resolver analiticamente classes de
problemas para os quais a técnica de separação de variáveis não é adequada.
No caso específico das equações de difusão linear, as transformações integrais procuram reduzir a quantidade de variáveis independentes
na equação original, geralmente levando a uma nova equação diferencial ordinária de fácil solução. Exemplos elementares de aplicação
de transformação integral em problemas difusivos nos sistemas cartesiano e cilíndrico foram apresentados por \cite{artigo_doetsch}, \cite{artigo_sneddon} e \cite{livro_tranter},
e em coordenadas esféricas por \cite{artigo_olcer}.
\cite{livro_unified} classificaram e revisaram sete classes de problemas difusivos lineares, apresentando suas soluções
exatas obtidas pela técnica da transformação integral, que por razões históricas passou a ser denominada Técnica da Transformação Integral Clássica (CITT).

A CITT não se mostrou flexível o suficiente para fornecer
soluções analíticas para tipos de problemas mais gerais. Nesse sentido, o trabalho desenvolvido por \cite{artigo_murray},
que tratava problemas difusivos transientes com coeficientes variáveis nas condições de contorno, foi pioneiro no desenvolvimento da técnica
que é conhecida hoje como Transformada Integral Generalizada (GITT) \citep{livro_integral_transforms_cotta}. Desde então, a GITT --- uma técnica
híbrida analítico-numérica --- vem sido continuamente desenvolvida e estendida, fornecendo soluções analíticas aproximadas e numéricas alternativas
para os problemas não solucionáveis pela abordagem clássica. Nos últimos anos, a GITT tem sido beneficiada e popularizada através do desenvolvimento de
aplicativos de \textit{software} de computação simbólica, especialmente o \textit{Mathematica}\textsuperscript{\textregistered} \citep{artigo_mathematica}.

A técnica de transformação integral, seja clássica ou generalizada, é formulada com base em integrais sobre um domínio cuja
fronteira não é necessariamente regular. Em muitos casos, a transformação integral foi empregada em problemas difusivos ou advectivo-difusivos nos quais a geometria
era irregular, porém descrita de forma algébrica no sistema de coordenadas cartesiano, como por exemplo os trabalhos de \cite{artigo_aparecido, artigo_aparecido_2, artigo_aparecido_3, artigo_fausto} e \cite{artigo_perez}.
Já \cite{tese_sphaier} generaliza as técnicas de levantamento de autofunções nestes trabalhos e apresenta uma solução formal para domínios irregulares usando GITT;
para fronteiras cuja descrição algébrica é desconhecida, Sphaier sugere aproximar por linhas poligonais (no caso bidimensional), em que cada
trecho é representado por um segmento de reta. As autofunções levantadas no processo de solução são obtidas a partir de problemas auxiliares resolvidos no próprio
domínio irregular do problema, e nesse caso diz-se que o domínio é \textit{coincidente}; é feita uma rápida citação sobre soluções de problemas difusivos em domínios \textit{envolventes}
(quando o domínio original é considerado como sendo contido por um domínio maior, regular, sobre o qual a solução geral é encontrada e então particularizada
para aquele subdomínio).

As soluções da equação de difusão em dutos de seção reta elíptica com condições de contorno de primeiro e segundo tipo encontradas por
 \citet{trabalho_maia_1, trabalho_maia_2} combinaram as técnicas de transformação integral e mapeamento conforme. 
Isto se seguiu com \cite{tese_antonini}, que usa a mesma metodologia em sua dissertação de mestrado sobre resolução de problemas difusivo-convectivos em geometrias não-convencionais (setor circular, geometria anular
concêntrica e geometria bicônica). Conforme citado anteriormente, as equações de difusão no novo sistema de coordenadas preservaram sua forma; isso facilitou em
muito o emprego da GITT, uma vez que os problemas de autovalor associado geraram funções seno e cosseno. Por outro lado, a
simplificação nas soluções analíticas ou híbridas encontradas só foi possível porque as geometrias envolvidas possuíam mapeamentos conformes de expressão analítica
conhecida \citep{livro_brown}.

Recentemente, houve o aumento do interesse na aplicação de métodos numéricos \textit{meshless} na resolução
de problemas difusivos em geometrias irregulares; tais métodos procuram superar as dificuldades computacionais envolvidas na geração
de malhas de discretização, trabalhando com o conceito de nós distribuídos pelo domínio do problema, e com as interações entre nós
vizinhos. Tais técnicas incluem, por exemplo, o emprego de redes neurais \citep{artigo_deng, artigo_heidari}, funções de base
radial associadas ao método de colocação \citep{artigo_chen, artigo_dai}, o método SPH (\textit{smoothed particle hydrodynamics}) \citep{artigo_vishwakarma} e o método MLPG (\textit{Meshless Local Petrov-Galerkin})
\citep{artigo_li}.

